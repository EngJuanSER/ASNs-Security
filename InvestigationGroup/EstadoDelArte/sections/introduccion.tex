\section{Introducción}

La evaluación de la legitimidad y seguridad de direcciones IP se ha convertido en una actividad fundamental para profesionales de ciberseguridad, administradores de sistemas y usuarios técnicos en el contexto actual de amenazas cibernéticas sofisticadas [1]. La creciente complejidad de los ataques cibernéticos y la fragmentación de información de seguridad en múltiples plataformas han impulsado el desarrollo de herramientas unificadas de diagnóstico que integren múltiples fuentes de datos para proporcionar evaluaciones comprehensivas de seguridad [2].

Este proyecto consolida, en un único flujo de consulta y presentación, información pasiva sobre direcciones IP usando exclusivamente fuentes gratuitas y de acceso público: (i) conjuntos de datos de Censys disponibles en Google BigQuery y (ii) la base GeoLite2 de MaxMind. No se usarán APIs con planes limitados ni servicios propietarios de pago.

\subsection{Propósito}

Aprovechar estas dos fuentes abiertas para producir una vista integrada que incluya, como mínimo: presencia observada en escaneos de Censys (puertos/servicios y certificados asociados) y geolocalización aproximada (país/ciudad si aplica) desde GeoLite2, presentadas de forma clara y reproducible.

\subsection{Alcance}

El trabajo comprende: (a) consulta y filtrado de datos en BigQuery (tablas públicas de Censys), (b) resolución local con GeoLite2, y (c) una plantilla de reporte que estandariza la lectura de resultados. No se crean nuevas fuentes ni métricas propietarias; se organiza y documenta lo existente.

\subsection{Restricciones}

- Solo se emplean fuentes gratuitas (BigQuery con datasets públicos de Censys y GeoLite2)
- Geolocalización es aproximada y sujeta a las limitaciones de la base
- No se ejecutan escaneos activos; se usa reconocimiento pasivo

\subsection{Entregables}

- Consultas de ejemplo en BigQuery sobre datasets de Censys
- Procedimiento para resolución/localización con GeoLite2
- Estructura de reporte para presentar hallazgos de manera consistente
