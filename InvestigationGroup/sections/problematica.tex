\section{Problemática}

\subsection{Necesidad y contexto}

La información relevante para contextualizar una dirección IP existe, pero suele estar fragmentada entre múltiples servicios y modelos de acceso. Parte importante del ecosistema impone límites de uso por API o requiere suscripciones, lo que afecta reproducibilidad y acceso.

\subsection{Criterio de selección de fuentes}

Para garantizar gratuidad, acceso sostenible y replicabilidad, este proyecto se restringe a:

\begin{itemize}
\item \textbf{Censys en Google BigQuery}: conjuntos de datos públicos que permiten consultas reproducibles sobre observaciones de puertos/servicios y certificados sin depender de claves de API
\item \textbf{MaxMind GeoLite2}: base gratuita ampliamente utilizada para geolocalización aproximada de IP
\end{itemize}

\subsection{Limitaciones asumidas}

\begin{itemize}
\item La geolocalización es aproximada (propia de GeoLite2)
\item Las observaciones de Censys reflejan cortes/ventanas de tiempo de los datasets públicos
\item Sin escaneo activo ni enriquecimientos propietarios
\end{itemize}

\subsection{Objetivo práctico}

Reducir la fricción operativa mediante una presentación integrada y documentada de estas dos fuentes, priorizando claridad, trazabilidad y facilidad de replicación sin costos.
