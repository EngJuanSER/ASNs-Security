\section{Cronograma de Trabajo}

\subsection{Duración y Alcance del Proyecto}

\textbf{Período Total:} 8 semanas (del 10 de octubre al 1 de diciembre de 2025)\\
\textbf{Equipo:} 2 estudiantes universitarios\\
\textbf{Modalidad:} Desarrollo ágil con entregas incrementales

\subsection{Metodología de Planificación}

El cronograma se estructura utilizando sprints cortos de 1 semana para permitir entregas rápidas y ajustes continuos. Cada sprint incluye desarrollo, testing básico y documentación del progreso. Se consideran las limitaciones de tiempo académico y la curva de aprendizaje para las tecnologías seleccionadas.

\subsection{Estructura de Desglose del Trabajo}

\subsubsection{Semana 1: Configuración e Investigación Inicial (10-17 Octubre)}
\textbf{Objetivos:} Establecer ambiente de desarrollo y comprender fuentes de datos

\textbf{Actividades principales:}
\begin{itemize}
    \item Configuración de entorno de desarrollo (Git, IDEs, cuentas cloud)
    \item Investigación de APIs de Censys y formato de datos BigQuery
    \item Descarga e instalación de GeoLite2
    \item Creación de prototipos de consulta básicos
    \item Definición de arquitectura inicial del sistema
\end{itemize}

\textbf{Entregables:}
\begin{itemize}
    \item Repositorio Git configurado
    \item Documentación de APIs disponibles
    \item Prototipo funcional de consulta a GeoLite2
    \item Diagrama de arquitectura básica
\end{itemize}

\subsubsection{Semana 2: Backend Base (17-24 Octubre)}
\textbf{Objetivos:} Implementar estructura básica del backend con Quarkus

\textbf{Actividades principales:}
\begin{itemize}
    \item Configuración inicial del proyecto Quarkus
    \item Implementación de API REST básica
    \item Integración con base de datos GeoLite2
    \item Desarrollo de servicios de geolocalización
    \item Testing unitario de componentes básicos
\end{itemize}

\textbf{Entregables:}
\begin{itemize}
    \item API REST funcional para geolocalización
    \item Documentación de endpoints
    \item Suite de tests unitarios básicos
    \item Dockerfile para contenedorización
\end{itemize}

\subsubsection{Semana 3: Integración BigQuery (24-31 Octubre)}
\textbf{Objetivos:} Conectar con datos de Censys a través de BigQuery

\textbf{Actividades principales:}
\begin{itemize}
    \item Configuración de acceso a Google Cloud Platform
    \item Implementación de cliente BigQuery
    \item Desarrollo de consultas SQL para datos de Censys
    \item Integración con servicios existentes de geolocalización
    \item Implementación de cache básico para optimizar consultas
\end{itemize}

\textbf{Entregables:}
\begin{itemize}
    \item Servicio de threat intelligence funcional
    \item API integrada (geolocalización + threat intelligence)
    \item Sistema de cache implementado
    \item Documentación de consultas BigQuery
\end{itemize}

\subsubsection{Semana 4: Frontend Inicial (31 Octubre - 7 Noviembre)}
\textbf{Objetivos:} Desarrollar interfaz básica con Vue.js

\textbf{Actividades principales:}
\begin{itemize}
    \item Configuración del proyecto Vue.js
    \item Desarrollo de componentes básicos (formularios, tablas)
    \item Implementación de cliente HTTP para APIs
    \item Desarrollo de vistas principales (búsqueda, resultados)
    \item Integración con mapas usando Leaflet
\end{itemize}

\textbf{Entregables:}
\begin{itemize}
    \item Aplicación web funcional básica
    \item Formulario de búsqueda de IP
    \item Visualización de resultados en tabla y mapa
    \item Conexión funcional con backend
\end{itemize}

\subsubsection{Semana 5: Funcionalidades Avanzadas (7-14 Noviembre)}
\textbf{Objetivos:} Mejorar funcionalidades y experiencia de usuario

\textbf{Actividades principales:}
\begin{itemize}
    \item Implementación de búsqueda por lotes (múltiples IPs)
    \item Desarrollo de funcionalidad de exportación (JSON, CSV)
    \item Mejora de visualizaciones (gráficos, mapas interactivos)
    \item Implementación de historial de búsquedas (local storage)
    \item Optimización de rendimiento frontend
\end{itemize}

\textbf{Entregables:}
\begin{itemize}
    \item Búsqueda masiva implementada
    \item Exportación de datos funcional
    \item Visualizaciones mejoradas
    \item Interfaz más intuitiva y responsiva
\end{itemize}

\subsubsection{Semana 6: Testing y Optimización (14-21 Noviembre)}
\textbf{Objetivos:} Validar funcionalidad y optimizar rendimiento

\textbf{Actividades principales:}
\begin{itemize}
    \item Testing funcional completo de la aplicación
    \item Pruebas de rendimiento con múltiples consultas
    \item Testing de usabilidad con usuarios del entorno académico
    \item Optimización de consultas y cache
    \item Corrección de bugs identificados
\end{itemize}

\textbf{Entregables:}
\begin{itemize}
    \item Suite completa de tests
    \item Reporte de testing de rendimiento
    \item Feedback de usuarios registrado
    \item Sistema optimizado y libre de bugs críticos
\end{itemize}

\subsubsection{Semana 7: Comparación y Validación (21-28 Noviembre)}
\textbf{Objetivos:} Validar resultados contra herramientas existentes

\textbf{Actividades principales:}
\begin{itemize}
    \item Comparación de resultados con herramientas públicas disponibles
    \item Análisis de precisión de geolocalización
    \item Evaluación de cobertura de threat intelligence
    \item Documentación de limitaciones encontradas
    \item Preparación de datos para presentación final
\end{itemize}

\textbf{Entregables:}
\begin{itemize}
    \item Reporte de comparación con herramientas existentes
    \item Análisis de precisión y limitaciones
    \item Dataset de validación documentado
    \item Métricas de rendimiento consolidadas
\end{itemize}

\subsubsection{Semana 8: Documentación y Entrega (28 Noviembre - 1 Diciembre)}
\textbf{Objetivos:} Completar documentación y preparar entrega final

\textbf{Actividades principales:}
\begin{itemize}
    \item Completar documentación técnica del código
    \item Elaborar manual de usuario
    \item Preparar guía de instalación y configuración
    \item Creación de presentación final
    \item Deployment en ambiente de producción (si es posible)
\end{itemize}

\textbf{Entregables:}
\begin{itemize}
    \item Documentación técnica completa
    \item Manual de usuario
    \item Guía de instalación
    \item Presentación final del proyecto
    \item Código fuente en repositorio público
\end{itemize}

\subsection{Cronograma Visual}

\begin{table}[H]
    \centering
    \small
    \begin{tabular}{|l|c|c|c|c|c|c|c|c|}
        \hline
        \textbf{Actividad} & \textbf{S1} & \textbf{S2} & \textbf{S3} & \textbf{S4} & \textbf{S5} & \textbf{S6} & \textbf{S7} & \textbf{S8} \\
        \hline
        Configuración inicial & X & & & & & & & \\
        \hline
        Backend desarrollo & & X & X & & & & & \\
        \hline
        Frontend desarrollo & & & & X & X & & & \\
        \hline
        Testing y optimización & & & & & & X & & \\
        \hline
        Validación y comparación & & & & & & & X & \\
        \hline
        Documentación final & & & & & & & & X \\
        \hline
    \end{tabular}
    \caption{Cronograma Semanal del Proyecto}
    \label{tab:cronograma_semanal}
\end{table}

\begin{table}[H]
    \centering
    \small
    \begin{tabular}{|c|l|l|c|}
        \hline
        \textbf{Hito} & \textbf{Descripción} & \textbf{Entregables} & \textbf{Fecha} \\
        \hline
        H1 & Configuración Completa & Ambiente de desarrollo, prototipos & Oct 17 \\
        \hline
        H2 & Backend Funcional & APIs operativas, integración datos & Oct 31 \\
        \hline
        H3 & Aplicación Completa & Frontend integrado, funcionalidades básicas & Nov 14 \\
        \hline
        H4 & Sistema Validado & Testing completo, optimizaciones & Nov 21 \\
        \hline
        H5 & Proyecto Finalizado & Documentación, entrega final & Dic 1 \\
        \hline
    \end{tabular}
    \caption{Hitos Principales del Proyecto}
    \label{tab:hitos}
\end{table}

\begin{table}[H]
    \centering
    \small
    \begin{tabular}{|l|c|c|l|}
        \hline
        \textbf{Riesgo} & \textbf{Probabilidad} & \textbf{Impacto} & \textbf{Mitigación} \\
        \hline
        Dificultades con BigQuery & Media & Alto & Tutoriales, soporte académico \\
        \hline
        Problemas de rendimiento & Alta & Medio & Testing temprano, optimización \\
        \hline
        Complejidad de Vue.js & Media & Medio & Documentación, ejemplos simples \\
        \hline
        Limitaciones de datos gratuitos & Baja & Alto & Fuentes alternativas, scope adjustment \\
        \hline
        Retrasos por carga académica & Alta & Medio & Planificación flexible, buffer time \\
        \hline
    \end{tabular}
    \caption{Matriz de Riesgos del Proyecto}
    \label{tab:riesgos}
\end{table}

\subsection{Recursos y Herramientas}

\subsubsection{Equipo de Trabajo}
\begin{itemize}
    \item \textbf{Estudiante 1:} Enfoque en backend y integración de datos
    \item \textbf{Estudiante 2:} Enfoque en frontend y experiencia de usuario
    \item \textbf{Colaboración:} Testing, documentación y validación conjunta
\end{itemize}

\subsubsection{Infraestructura Tecnológica}
\begin{itemize}
    \item \textbf{Desarrollo:} Computadoras personales con IDEs gratuitos
    \item \textbf{Cloud Services:} Google Cloud Platform (créditos académicos)
    \item \textbf{Repositorio:} GitHub (cuenta gratuita)
    \item \textbf{Comunicación:} Discord, WhatsApp, reuniones presenciales
    \item \textbf{Documentación:} Google Docs, LaTeX para documentos formales
\end{itemize}

\subsubsection{Herramientas de Software}
\begin{itemize}
    \item \textbf{Backend:} Quarkus, Java 21, PostgreSQL
    \item \textbf{Frontend:} Vue.js 3, Vuetify, Leaflet.js
    \item \textbf{Testing:} JUnit, Jest, Postman
    \item \textbf{Deployment:} Docker, Google Cloud Run
\end{itemize}
