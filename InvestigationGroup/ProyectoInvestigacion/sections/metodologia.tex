\section{Aspectos Metodológicos}

\subsection{Tipo de Estudio}

\subsubsection{Diseño de Investigación}
El estudio implementa un \textbf{diseño de investigación aplicada} con enfoque práctico-experimental. Se centra en el desarrollo e implementación de una herramienta funcional que integre fuentes de datos públicas para análisis de direcciones IP.

\textbf{Componentes del Estudio:}
\begin{itemize}
    \item \textbf{Desarrollo tecnológico:} Construcción de aplicación web funcional
    \item \textbf{Evaluación funcional:} Pruebas de rendimiento y usabilidad
    \item \textbf{Análisis comparativo:} Validación contra herramientas existentes
    \item \textbf{Documentación:} Registro sistemático del proceso y resultados
\end{itemize}

\subsection{Método de Investigación}

\subsubsection{Metodología de Desarrollo}
Se utilizará una metodología ágil adaptada al contexto académico, siguiendo principios de desarrollo iterativo ampliamente utilizados en proyectos de investigación aplicada~\cite{ferrag2019deep}:

\textbf{Desarrollo Iterativo:}
\begin{itemize}
    \item Sprints de 1 semana para desarrollo rápido
    \item Entregas incrementales con funcionalidades básicas
    \item Pruebas continuas durante el desarrollo
    \item Documentación paralela al código
\end{itemize}

\textbf{Arquitectura Orientada a Servicios:}
\begin{itemize}
    \item Separación clara entre frontend y backend
    \item APIs REST para comunicación entre componentes
    \item Integración modular con fuentes de datos externas
\end{itemize}

\subsection{Fuentes y Técnicas para Recolección de Información}

\subsubsection{Fuentes de Datos Primarias}

\textbf{Datasets Técnicos:}
\begin{enumerate}
    \item \textbf{Nmap (Network Mapper):}
    \begin{itemize}
        \item Herramienta de escaneo de red de código abierto~\cite{lyon2009nmap}
        \item Detección de servicios activos y versiones de software
        \item Configuración: TCP connect scan (-sT) en lugar de SYN scan para compatibilidad
        \item Salida en formato XML para procesamiento automatizado
        \item Comando: \texttt{nmap -sT -sV -p [ports] --open -T4 -oX [output] [target]}
        \item Validada como herramienta estándar en estudios comparativos~\cite{pittman2023comparative}
    \end{itemize}
    
    \item \textbf{NVD API (National Vulnerability Database):}
    \begin{itemize}
        \item API REST pública de NIST para consulta de CVEs
        \item Endpoint: https://services.nvd.nist.gov/rest/json/cves/2.0
        \item Búsqueda por CPE (Common Platform Enumeration)
        \item Información detallada de severidad mediante CVSS v3.1
        \item Modo gratuito con rate limiting natural (5 requests/30s)
        \item Sin necesidad de API key para consultas individuales
        \item Utilizada en investigaciones de inteligencia de amenazas~\cite{lin2023correlation}
    \end{itemize}
    
    \item \textbf{MaxMind GeoLite2:}
    \begin{itemize}
        \item Base de datos descargable en formato MMDB (61MB comprimido)
        \item Fuente: GitHub repository P3TERX/GeoLite.mmdb (actualizada mensualmente)
        \item Descarga automática durante Docker build: \texttt{curl -L [url] -o geo/GeoLite2-City.mmdb}
        \item Almacenamiento local en \texttt{src/main/resources/geo/} del backend
        \item Precisión validada en contextos académicos~\cite{schopman2021validating}
        \item Incluye: país, región, ciudad, coordenadas, timezone, ASN, ISP
    \end{itemize}
    
    \item \textbf{ipapi.co (API Complementaria):}
    \begin{itemize}
        \item API REST pública para información ASN e ISP actualizada
        \item Endpoint: https://ipapi.co/[ip]/json
        \item Complementa datos de GeoLite2 con información de red actual
        \item Rate limit: 1000 requests/día en modo gratuito
        \item Proporciona: ASN, ISP, organización, tipo de red
    \end{itemize}
\end{enumerate}

\subsubsection{Técnicas de Recolección}

\textbf{Integración Automatizada:}
\begin{itemize}
    \item \textbf{Process Execution:} Ejecución de Nmap desde Java mediante ProcessBuilder
    \item \textbf{XML Parsing:} Procesamiento de output XML de Nmap con JAXB o DOM
    \item \textbf{REST APIs:} Consultas programáticas a NVD API para obtener CVEs
    \item \textbf{File Processing:} Lectura local de bases de datos MMDB de GeoLite2
    \item \textbf{Caching:} Almacenamiento temporal para optimizar rendimiento
\end{itemize}

\textbf{Validación de Datos:}
\begin{itemize}
    \item Verificación de formato y consistencia de datos
    \item Filtrado de información irrelevante o duplicada
    \item Manejo de errores y datos faltantes
    \item Logging de actividades para auditoría
\end{itemize}

\subsection{Arquitectura del Sistema}

\subsubsection{Componentes Principales}

\textbf{Backend (Quarkus):}
\begin{itemize}
    \item \textbf{AnalysisResource:} Endpoint REST principal POST /api/analysis/analyze
    \item \textbf{AnalysisService:} Orquestador del flujo completo de análisis
    \item \textbf{GeolocationService:} Lectura de GeoLite2 MMDB con maxmind-db-reader
    \item \textbf{NmapService:} Ejecución de nmap via ProcessBuilder y parsing de XML
    \item \textbf{NVDService:} Cliente REST para consultas a NVD API con rate limiting
    \item \textbf{ASNService:} Consultas a ipapi.co para información de red actualizada
    \item \textbf{SecurityScoringService:} Cálculo de score de seguridad basado en múltiples factores
    \item \textbf{RecommendationService:} Sistema experto para generación de recomendaciones
    \item \textbf{Arquitectura:} Stateless sin persistencia, todas las consultas son efímeras
\end{itemize}

\textbf{Frontend (Vue.js):}
\begin{itemize}
    \item \textbf{Vista de Análisis (Analysis.vue):} Análisis individual con visualización completa
    \item \textbf{Vista de Comparación (IPComparator.vue):} Comparación lado a lado de dos targets
    \item \textbf{Vista de Historial (History.vue):} Historial con estadísticas y gráficos Chart.js
    \item \textbf{Sistema de Cache (LocalForage):} Almacenamiento local con TTL de 7 días
    \item \textbf{Componentes de Mapas:} Leaflet.js para visualización geográfica interactiva
\end{itemize}

\subsubsection{Flujo de Datos}

\begin{enumerate}
    \item \textbf{Entrada:} Usuario ingresa dirección IP o dominio a analizar
    \item \textbf{Validación:} Sistema verifica formato (IPv4, IPv6, o dominio)
    \item \textbf{Resolución DNS:} Si es dominio, resuelve a dirección IP mediante Java InetAddress
    \item \textbf{Verificación de Cache:} Frontend consulta LocalForage (TTL 7 días)
    \item \textbf{Procesamiento Paralelo en Backend:}
    \begin{itemize}
        \item Geolocalización via GeoLite2 (consulta local MMDB, $<$1ms)
        \item Consulta ASN/ISP via ipapi.co (complementa datos de red)
        \item Escaneo de servicios via Nmap (TCP connect scan -sT, 1-60s)
    \end{itemize}
    \item \textbf{Análisis de Vulnerabilidades:} 
    \begin{itemize}
        \item Construcción de CPEs desde servicios detectados
        \item Consulta NVD API por cada CPE
        \item Agregación de CVEs con metadatos (CVSS, severidad, referencias)
    \end{itemize}
    \item \textbf{Scoring de Seguridad:} Cálculo basado en: cantidad de vulnerabilidades, severidad CVSS máxima, puertos críticos expuestos, servicios desactualizados
    \item \textbf{Generación de Recomendaciones:} Sistema experto basado en reglas para sugerir mitigaciones
    \item \textbf{Presentación Frontend:} 
    \begin{itemize}
        \item Visualización unificada con componentes Vue
        \item Almacenamiento en cache LocalForage
        \item Registro en historial con statisticsService
    \end{itemize}
\end{enumerate}

\subsection{Herramientas y Tecnologías}

\subsubsection{Stack de Desarrollo}

\textbf{Backend - Quarkus Framework:}
\begin{itemize}
    \item Lenguaje: Java 21 LTS (eclipse-temurin:21)
    \item Framework: Quarkus 3.29.2 con configuración optimizada
    \item Arquitectura: Stateless sin base de datos persistente
    \item APIs: REST con Jackson para serialización JSON
    \item Build: Gradle 9.0 con multi-stage Docker build
\end{itemize}

\textbf{Frontend - Vue.js Ecosystem:}
\begin{itemize}
    \item Framework: Vue.js 3 con Composition API y TypeScript 5
    \item Build Tool: Vite 6 para desarrollo rápido y optimización
    \item Routing: Vue Router para navegación SPA
    \item Cache: LocalForage (IndexedDB) con TTL de 7 días
    \item Visualizaciones: Chart.js para gráficos, Leaflet.js para mapas
    \item UI: CSS personalizado con sistema de temas (claro/oscuro)
\end{itemize}

\textbf{Infraestructura y Despliegue:}
\begin{itemize}
    \item Backend: Railway (https://asns-security-production.up.railway.app)
    \item Frontend: Vercel (https://diagsec.vercel.app)
    \item Contenedores: Docker multi-stage con Alpine Linux
    \item Nmap: Preinstalado con TCP connect scan (-sT) para compatibilidad Railway
    \item CI/CD: Despliegue automático desde GitHub (main branch)
    \item Variables de entorno: CORS\_ORIGINS, VITE\_API\_BASE\_URL, QUARKUS\_PROFILE
\end{itemize}

\subsubsection{Herramientas de Desarrollo}

\textbf{IDEs y Editores:}
\begin{itemize}
    \item IntelliJ IDEA Community para Java/Quarkus
    \item Visual Studio Code para Vue.js
    \item Postman para testing de APIs
\end{itemize}

\textbf{Control de Versiones:}
\begin{itemize}
    \item Git con GitHub para repositorio
    \item Branching strategy simplificado
    \item Issues tracking para gestión de tareas
\end{itemize}

\subsection{Validación y Testing}

\subsubsection{Estrategias de Prueba}

\textbf{Testing Funcional:}
\begin{itemize}
    \item Unit tests para componentes individuales
    \item Integration tests para APIs
    \item End-to-end tests para flujos completos
    \item Manual testing de interfaz de usuario
\end{itemize}

\textbf{Testing de Rendimiento:}
\begin{itemize}
    \item Pruebas de carga con múltiples consultas simultáneas
    \item Medición de tiempos de respuesta
    \item Evaluación de uso de memoria y CPU
    \item Testing de tiempos de ejecución de Nmap
\end{itemize}

\subsubsection{Validación de Resultados}

\textbf{Comparación con Herramientas Existentes:}
\begin{itemize}
    \item Verificación de geolocalización contra servicios conocidos
    \item Comparación de detección de servicios con Nmap directo
    \item Validación de CVEs contra base de datos NVD oficial
    \item Evaluación de precisión y cobertura de datos
\end{itemize}

\textbf{Testing de Usuario:}
\begin{itemize}
    \item Pruebas de usabilidad con estudiantes y profesores
    \item Evaluación de intuitividad de la interfaz
    \item Recolección de feedback para mejoras
\end{itemize}

\subsection{Consideraciones Éticas y Técnicas}

\subsubsection{Uso Responsable de Datos}
\begin{itemize}
    \item Uso exclusivo de datos públicos y gratuitos (GeoLite2, NVD, ipapi.co)
    \item TCP connect scan (-sT) en lugar de SYN scan (menos intrusivo, no requiere root)
    \item No almacenamiento persistente en backend (arquitectura stateless)
    \item Cache frontend limitado a 7 días con limpieza automática
    \item Rate limiting natural de APIs externas respetado (NVD: 5/30s, ipapi: 1000/día)
    \item URLs públicas desplegadas: https://diagsec.vercel.app (frontend) y https://asns-security-production.up.railway.app (backend)
\end{itemize}

\subsubsection{Limitaciones Técnicas Reconocidas}
\begin{itemize}
    \item \textbf{Escaneo Nmap:} TCP connect scan (-sT) es 20-30\% más lento que SYN scan (-sS), pero compatible con Railway que bloquea raw sockets
    \item \textbf{Tiempo de ejecución:} Escaneos varían de 1-60 segundos según cantidad de puertos y red objetivo
    \item \textbf{GeoLite2:} Base de datos de 61MB descargada durante Docker build, actualización manual mensual requerida
    \item \textbf{NVD API:} Rate limiting natural de 5 requests/30s en modo gratuito (sin API key), suficiente para análisis individuales
    \item Dependencia de disponibilidad de APIs externas (NVD, ipapi.co)
    \item Precisión de geolocalización variable según región (95\% país, 50-75\% ciudad)
    \item Capacidad de procesamiento limitada por plan gratuito de Railway
    \item Sin persistencia de datos históricos en backend (solo cache temporal frontend)
\end{itemize}
    \item Dependencia de disponibilidad de NVD API
    \item Precisión limitada por calidad de fuentes de datos
    \item Capacidad de procesamiento limitada a recursos académicos
    \item Cobertura geográfica variable según la fuente
\end{itemize}

