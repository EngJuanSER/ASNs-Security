\section{Aspectos Metodológicos}

\subsection{Tipo de Estudio}

\subsubsection{Diseño de Investigación}
El estudio implementa un \textbf{diseño de investigación aplicada} con enfoque práctico-experimental. Se centra en el desarrollo e implementación de una herramienta funcional que integre fuentes de datos públicas para análisis de direcciones IP.

\textbf{Componentes del Estudio:}
\begin{itemize}
    \item \textbf{Desarrollo tecnológico:} Construcción de aplicación web funcional
    \item \textbf{Evaluación funcional:} Pruebas de rendimiento y usabilidad
    \item \textbf{Análisis comparativo:} Validación contra herramientas existentes
    \item \textbf{Documentación:} Registro sistemático del proceso y resultados
\end{itemize}

\subsection{Método de Investigación}

\subsubsection{Metodología de Desarrollo}
Se propone utilizar una metodología ágil adaptada al contexto académico, siguiendo principios de desarrollo iterativo ampliamente utilizados en proyectos de investigación aplicada~\cite{ferrag2019deep}:

\textbf{Desarrollo Iterativo:}
\begin{itemize}
    \item Sprints de 1 semana para desarrollo rápido
    \item Entregas incrementales con funcionalidades básicas
    \item Pruebas continuas durante el desarrollo
    \item Documentación paralela al código
\end{itemize}

\textbf{Arquitectura Orientada a Servicios:}
\begin{itemize}
    \item Separación clara entre frontend y backend
    \item APIs REST para comunicación entre componentes
    \item Integración modular con fuentes de datos externas
\end{itemize}

\subsection{Fuentes y Técnicas para Recolección de Información}

\subsubsection{Fuentes de Datos Primarias}

\textbf{Datasets Técnicos:}
\begin{enumerate}
    \item \textbf{Nmap (Network Mapper):}
    \begin{itemize}
        \item Herramienta de escaneo de red de código abierto~\cite{lyon2009nmap}
        \item Detección de servicios activos y versiones de software
        \item Configuración: TCP connect scan (-sT) en lugar de SYN scan para compatibilidad
        \item Salida en formato XML para procesamiento automatizado
        \item Comando: \texttt{nmap -sT -sV -p [ports] --open -T4 -oX [output] [target]}
        \item Validada como herramienta estándar en estudios comparativos~\cite{pittman2023comparative}
    \end{itemize}
    
    \item \textbf{NVD API (National Vulnerability Database):}
    \begin{itemize}
        \item API REST pública de NIST para consulta de CVEs
        \item Endpoint: https://services.nvd.nist.gov/rest/json/cves/2.0
        \item Búsqueda por CPE (Common Platform Enumeration)
        \item Información detallada de severidad mediante CVSS v3.1
        \item Modo gratuito con rate limiting natural (5 requests/30s)
        \item Sin necesidad de API key para consultas individuales
        \item Utilizada en investigaciones de inteligencia de amenazas~\cite{lin2023correlation}
    \end{itemize}
    
    \item \textbf{MaxMind GeoLite2:}
    \begin{itemize}
        \item Base de datos descargable en formato MMDB (aproximadamente 60MB)
        \item Disponible en repositorios públicos de GitHub
        \item Descarga automatizable durante proceso de build
        \item Almacenamiento local en directorio de recursos del backend
        \item Precisión validada en contextos académicos~\cite{schopman2021validating}
        \item Incluye: país, región, ciudad, coordenadas, timezone, ASN, ISP
    \end{itemize}
    
    \item \textbf{ipapi.co (API Complementaria):}
    \begin{itemize}
        \item API REST pública para información ASN e ISP actualizada
        \item Endpoint: https://ipapi.co/[ip]/json
        \item Complementa datos de GeoLite2 con información de red actual
        \item Rate limit: 1000 requests/día en modo gratuito
        \item Proporciona: ASN, ISP, organización, tipo de red
    \end{itemize}
\end{enumerate}

\subsubsection{Técnicas de Recolección}

\textbf{Integración Automatizada:}
\begin{itemize}
    \item \textbf{Process Execution:} Ejecución de Nmap desde el lenguaje de backend mediante APIs de procesos
    \item \textbf{XML Parsing:} Procesamiento de output XML de Nmap con librerías estándar
    \item \textbf{REST APIs:} Consultas programáticas a NVD API para obtener CVEs
    \item \textbf{File Processing:} Lectura local de bases de datos MMDB de GeoLite2
    \item \textbf{Caching:} Almacenamiento temporal para optimizar rendimiento
\end{itemize}

\textbf{Validación de Datos:}
\begin{itemize}
    \item Verificación de formato y consistencia de datos
    \item Filtrado de información irrelevante o duplicada
    \item Manejo de errores y datos faltantes
    \item Logging de actividades para auditoría
\end{itemize}

\subsection{Arquitectura del Sistema}

\subsubsection{Componentes Principales}

\textbf{Backend:}
Se propone una arquitectura de servicios modulares donde cada componente tiene una responsabilidad específica:
\begin{itemize}
    \item \textbf{Controlador REST:} Endpoint principal para recibir solicitudes de análisis
    \item \textbf{Servicio Orquestador:} Coordina la ejecución del flujo completo de análisis
    \item \textbf{Servicio de DNS:} Resolución de dominios a direcciones IP
    \item \textbf{Servicio de Geolocalización:} Consulta a base de datos GeoLite2
    \item \textbf{Servicio de Escaneo:} Ejecución de Nmap y parsing de resultados
    \item \textbf{Servicio de Vulnerabilidades:} Consultas a NVD API
    \item \textbf{Servicio de ASN:} Consultas a ipapi.co para información de red
    \item \textbf{Servicio de Scoring:} Cálculo de puntuación de seguridad
    \item \textbf{Servicio de Recomendaciones:} Generación de sugerencias basadas en hallazgos
\end{itemize}

\textbf{Frontend:}
Se propone una aplicación de página única (SPA) con las siguientes vistas principales:
\begin{itemize}
    \item \textbf{Vista de Inicio:} Página principal con campo de búsqueda
    \item \textbf{Vista de Análisis:} Visualización completa de resultados del análisis
    \item \textbf{Vista de Comparación:} Comparación lado a lado de dos análisis
    \item \textbf{Vista de Historial:} Registro de análisis anteriores con estadísticas
    \item \textbf{Sistema de Cache:} Almacenamiento local para persistencia de datos
    \item \textbf{Componentes de Mapas:} Visualización geográfica interactiva
\end{itemize}

\subsubsection{Flujo de Datos}

\begin{enumerate}
    \item \textbf{Entrada:} Usuario ingresa dirección IP o dominio a analizar
    \item \textbf{Validación:} Sistema verifica formato (IPv4, IPv6, o dominio)
    \item \textbf{Resolución DNS:} Si es dominio, resuelve a dirección IP mediante APIs estándar del lenguaje
    \item \textbf{Verificación de Cache:} Frontend consulta LocalForage para resultados previos
    \item \textbf{Procesamiento Paralelo en Backend:}
    \begin{itemize}
        \item Geolocalización via GeoLite2 (consulta local MMDB, $<$1ms)
        \item Consulta ASN/ISP via ipapi.co (complementa datos de red)
        \item Escaneo de servicios via Nmap (TCP connect scan -sT, 1-60s)
    \end{itemize}
    \item \textbf{Análisis de Vulnerabilidades:} 
    \begin{itemize}
        \item Construcción de CPEs desde servicios detectados
        \item Consulta NVD API por cada CPE
        \item Agregación de CVEs con metadatos (CVSS, severidad, referencias)
    \end{itemize}
    \item \textbf{Scoring de Seguridad:} Cálculo mediante sistema de penalizaciones desde un score base de 100, considerando:
    \begin{itemize}
        \item Penalizaciones por puertos de alto riesgo (Telnet, FTP, RDP, VNC, SMB)
        \item Penalizaciones por servicios administrativos expuestos (bases de datos, SSH)
        \item Penalizaciones por vulnerabilidades según severidad CVSS (CRITICAL, HIGH, MEDIUM, LOW)
        \item Penalizaciones por superficie de ataque excesiva
        \item Clasificación final: LOW ($\geq$80), MEDIUM ($\geq$60), HIGH ($<$60)
    \end{itemize}
    \item \textbf{Generación de Recomendaciones:} Sistema basado en reglas para sugerir mitigaciones según vulnerabilidades detectadas, protocolos inseguros y servicios expuestos.
    \item \textbf{Presentación Frontend:} 
    \begin{itemize}
        \item Visualización unificada en interfaz web
        \item Almacenamiento en cache local
        \item Registro en historial de análisis
    \end{itemize}
\end{enumerate}

\subsection{Herramientas y Tecnologías}

\subsubsection{Stack de Desarrollo Propuesto}

\textbf{Backend:}
\begin{itemize}
    \item Lenguaje: Java (versión LTS) por robustez y ecosistema maduro
    \item Framework: Quarkus, optimizado para aplicaciones cloud-native
    \item Arquitectura: API REST stateless para facilitar despliegue y escalabilidad
    \item Build: Gradle con soporte para contenedores Docker
\end{itemize}

\textbf{Frontend:}
\begin{itemize}
    \item Framework: Vue.js 3 con Composition API
    \item Lenguaje: TypeScript para tipado estático
    \item Build Tool: Vite para desarrollo rápido
    \item Cache: LocalForage (IndexedDB) para persistencia local
    \item Visualizaciones: Chart.js para gráficos, Leaflet.js para mapas geográficos
\end{itemize}

\textbf{Infraestructura:}
\begin{itemize}
    \item Contenedores: Docker para empaquetado y despliegue consistente
    \item Hosting: Plataformas cloud con tier gratuito
    \item CI/CD: Integración con repositorio Git para despliegue automático
\end{itemize}

\subsubsection{Herramientas de Desarrollo}

\textbf{IDEs y Editores:}
\begin{itemize}
    \item IDEs apropiados para el lenguaje de backend seleccionado
    \item Editor de código para el desarrollo frontend
    \item Herramientas de testing de APIs (Postman o similares)
\end{itemize}

\textbf{Control de Versiones:}
\begin{itemize}
    \item Git con GitHub para repositorio
    \item Branching strategy simplificado
    \item Issues tracking para gestión de tareas
\end{itemize}

\subsection{Validación y Testing}

\subsubsection{Estrategias de Prueba}

\textbf{Testing Funcional:}
\begin{itemize}
    \item Unit tests para componentes individuales
    \item Integration tests para APIs
    \item End-to-end tests para flujos completos
    \item Manual testing de interfaz de usuario
\end{itemize}

\textbf{Testing de Rendimiento:}
\begin{itemize}
    \item Pruebas de carga con múltiples consultas simultáneas
    \item Medición de tiempos de respuesta
    \item Evaluación de uso de memoria y CPU
    \item Testing de tiempos de ejecución de Nmap
\end{itemize}

\subsubsection{Validación de Resultados}

\textbf{Comparación con Herramientas Existentes:}
\begin{itemize}
    \item Verificación de geolocalización contra servicios conocidos
    \item Comparación de detección de servicios con Nmap directo
    \item Validación de CVEs contra base de datos NVD oficial
    \item Evaluación de precisión y cobertura de datos
\end{itemize}

\textbf{Testing de Usuario:}
\begin{itemize}
    \item Pruebas de usabilidad con estudiantes y profesores
    \item Evaluación de intuitividad de la interfaz
    \item Recolección de feedback para mejoras
\end{itemize}

\subsection{Consideraciones Éticas y Técnicas}

\subsubsection{Uso Responsable de Datos}
\begin{itemize}
    \item Uso exclusivo de datos públicos y gratuitos (GeoLite2, NVD, ipapi.co)
    \item TCP connect scan en lugar de SYN scan (menos intrusivo, no requiere privilegios de root)
    \item Arquitectura stateless sin almacenamiento persistente en backend
    \item Historial de análisis con limpieza automática de datos antiguos
    \item Respeto a rate limits de APIs externas
\end{itemize}

\subsubsection{Limitaciones Técnicas Reconocidas}
\begin{itemize}
    \item \textbf{Escaneo Nmap:} TCP connect scan es más lento que SYN scan, pero compatible con entornos cloud que bloquean raw sockets
    \item \textbf{Tiempo de ejecución:} Los escaneos pueden variar significativamente según la red objetivo
    \item \textbf{GeoLite2:} La base de datos requiere actualizaciones periódicas
    \item \textbf{NVD API:} Rate limiting en modo gratuito puede afectar análisis masivos
    \item Dependencia de disponibilidad de APIs externas
    \item Precisión de geolocalización variable según región geográfica
    \item Capacidad de procesamiento limitada por planes gratuitos de hosting
    \item Sin persistencia de datos históricos en backend (solo cache en frontend)
\end{itemize}

