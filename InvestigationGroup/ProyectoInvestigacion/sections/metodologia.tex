\section{Aspectos Metodológicos}

\subsection{Tipo de Estudio}

\subsubsection{Diseño de Investigación}
El estudio implementa un \textbf{diseño de investigación aplicada} con enfoque práctico-experimental. Se centra en el desarrollo e implementación de una herramienta funcional que integre fuentes de datos públicas para análisis de direcciones IP.

\textbf{Componentes del Estudio:}
\begin{itemize}
    \item \textbf{Desarrollo tecnológico:} Construcción de aplicación web funcional
    \item \textbf{Evaluación funcional:} Pruebas de rendimiento y usabilidad
    \item \textbf{Análisis comparativo:} Validación contra herramientas existentes
    \item \textbf{Documentación:} Registro sistemático del proceso y resultados
\end{itemize}

\subsection{Método de Investigación}

\subsubsection{Metodología de Desarrollo}
Se utilizará una metodología ágil adaptada al contexto académico:

\textbf{Desarrollo Iterativo:}
\begin{itemize}
    \item Sprints de 1 semana para desarrollo rápido
    \item Entregas incrementales con funcionalidades básicas
    \item Pruebas continuas durante el desarrollo
    \item Documentación paralela al código
\end{itemize}

\textbf{Arquitectura Orientada a Servicios:}
\begin{itemize}
    \item Separación clara entre frontend y backend
    \item APIs REST para comunicación entre componentes
    \item Integración modular con fuentes de datos externas
\end{itemize}

\subsection{Fuentes y Técnicas para Recolección de Información}

\subsubsection{Fuentes de Datos Primarias}

\textbf{Datasets Técnicos:}
\begin{enumerate}
    \item \textbf{Censys via BigQuery:}
    \begin{itemize}
        \item Datos de escaneo de Internet actualizados semanalmente
        \item Información de puertos, servicios y certificados
        \item Acceso a través de consultas SQL estándar
        \item Quota gratuita de 1TB mensual para consultas
    \end{itemize}
    
    \item \textbf{MaxMind GeoLite2:}
    \begin{itemize}
        \item Base de datos descargable en formato MMDB
        \item Actualizaciones mensuales automáticas
        \item Cobertura global con precisión variable por región
        \item Licencia gratuita para uso académico
    \end{itemize}
\end{enumerate}

\subsubsection{Técnicas de Recolección}

\textbf{Integración Automatizada:}
\begin{itemize}
    \item \textbf{APIs REST:} Consultas programáticas a servicios externos
    \item \textbf{SQL Queries:} Extracción eficiente de datos desde BigQuery
    \item \textbf{File Processing:} Lectura local de bases de datos MMDB
    \item \textbf{Caching:} Almacenamiento temporal para optimizar rendimiento
\end{itemize}

\textbf{Validación de Datos:}
\begin{itemize}
    \item Verificación de formato y consistencia de datos
    \item Filtrado de información irrelevante o duplicada
    \item Manejo de errores y datos faltantes
    \item Logging de actividades para auditoría
\end{itemize}

\subsection{Arquitectura del Sistema}

\subsubsection{Componentes Principales}

\textbf{Backend (Quarkus):}
\begin{itemize}
    \item \textbf{API Gateway:} Punto de entrada único para solicitudes
    \item \textbf{Servicio de Geolocalización:} Integración con GeoLite2
    \item \textbf{Servicio de Threat Intelligence:} Consultas a BigQuery/Censys
    \item \textbf{Servicio de Correlación:} Combinación de datos de múltiples fuentes
    \item \textbf{Cache Service:} Almacenamiento temporal de resultados frecuentes
\end{itemize}

\textbf{Frontend (Vue.js):}
\begin{itemize}
    \item \textbf{Interfaz de Consulta:} Formularios para ingreso de direcciones IP
    \item \textbf{Dashboard de Resultados:} Visualización de datos integrados
    \item \textbf{Componentes de Mapas:} Representación geográfica de resultados
    \item \textbf{Exportación de Datos:} Funcionalidades para guardar resultados
\end{itemize}

\subsubsection{Flujo de Datos}

\begin{enumerate}
    \item \textbf{Entrada:} Usuario ingresa dirección IP a analizar
    \item \textbf{Validación:} Sistema verifica formato y validez de la IP
    \item \textbf{Consulta Paralela:}
    \begin{itemize}
        \item Geolocalización via GeoLite2
        \item Threat intelligence via Censys/BigQuery
    \end{itemize}
    \item \textbf{Correlación:} Integración de resultados de múltiples fuentes
    \item \textbf{Presentación:} Visualización unificada en interfaz web
\end{enumerate}

\subsection{Herramientas y Tecnologías}

\subsubsection{Stack de Desarrollo}

\textbf{Backend - Quarkus Framework:}
\begin{itemize}
    \item Lenguaje: Java 17 LTS
    \item Framework: Quarkus 3.x
    \item Base de datos: PostgreSQL para cache y logs
    \item APIs: REST con Jackson para JSON
\end{itemize}

\textbf{Frontend - Vue.js Ecosystem:}
\begin{itemize}
    \item Framework: Vue.js 3 con Composition API
    \item Routing: Vue Router
    \item State Management: Pinia
    \item UI Components: Vuetify o Bootstrap Vue
    \item Mapas: Leaflet.js para visualización geográfica
\end{itemize}

\textbf{Infraestructura y Despliegue:}
\begin{itemize}
    \item Cloud Platform: Google Cloud Platform
    \item Contenedores: Docker para empaquetado
    \item CI/CD: GitHub Actions
    \item Monitoreo: Logs básicos y métricas de rendimiento
\end{itemize}

\subsubsection{Herramientas de Desarrollo}

\textbf{IDEs y Editores:}
\begin{itemize}
    \item IntelliJ IDEA Community para Java/Quarkus
    \item Visual Studio Code para Vue.js
    \item Postman para testing de APIs
\end{itemize}

\textbf{Control de Versiones:}
\begin{itemize}
    \item Git con GitHub para repositorio
    \item Branching strategy simplificado
    \item Issues tracking para gestión de tareas
\end{itemize}

\subsection{Validación y Testing}

\subsubsection{Estrategias de Prueba}

\textbf{Testing Funcional:}
\begin{itemize}
    \item Unit tests para componentes individuales
    \item Integration tests para APIs
    \item End-to-end tests para flujos completos
    \item Manual testing de interfaz de usuario
\end{itemize}

\textbf{Testing de Rendimiento:}
\begin{itemize}
    \item Pruebas de carga con múltiples consultas simultáneas
    \item Medición de tiempos de respuesta
    \item Evaluación de uso de memoria y CPU
    \item Testing de límites de quota de APIs externas
\end{itemize}

\subsubsection{Validación de Resultados}

\textbf{Comparación con Herramientas Existentes:}
\begin{itemize}
    \item Verificación de geolocalización contra servicios conocidos
    \item Comparación de threat intelligence con fuentes públicas
    \item Evaluación de precisión y cobertura de datos
\end{itemize}

\textbf{Testing de Usuario:}
\begin{itemize}
    \item Pruebas de usabilidad con estudiantes y profesores
    \item Evaluación de intuitividad de la interfaz
    \item Recolección de feedback para mejoras
\end{itemize}

\subsection{Consideraciones Éticas y Técnicas}

\subsubsection{Uso Responsable de Datos}
\begin{itemize}
    \item Uso exclusivo de datos públicos y gratuitos
    \item Respeto a términos de servicio de proveedores
    \item No almacenamiento persistente de consultas de usuarios
    \item Implementación de rate limiting para evitar abuso
\end{itemize}

\subsubsection{Limitaciones Técnicas Reconocidas}
\begin{itemize}
    \item Dependencia de disponibilidad de servicios externos
    \item Precisión limitada por calidad de fuentes de datos
    \item Capacidad de procesamiento limitada a recursos académicos
    \item Cobertura geográfica variable según la fuente
\end{itemize}
