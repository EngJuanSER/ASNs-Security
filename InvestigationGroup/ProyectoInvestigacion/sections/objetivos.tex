\section{Objetivos de la Investigación}

\subsection{Objetivo General}
Diseñar, desarrollar y documentar una herramienta de software funcional que integre datos de geolocalización de MaxMind GeoLite2, escaneo de servicios con Nmap (TCP connect scan), análisis de vulnerabilidades mediante NVD API, e información de red mediante ipapi.co, para presentar un perfil unificado y de fácil consulta sobre direcciones IP y dominios, desplegada en plataformas cloud (Railway + Vercel).

\subsection{Objetivos Específicos}

\begin{enumerate}
    \item \textbf{Diseñar una arquitectura de software cloud-native} para la herramienta, basada en un backend stateless (Quarkus 3.29.2 + Java 21) que centralice la lógica de negocio, desplegado en Railway, y un frontend SPA (Vue.js 3 + Vite + TypeScript) que consuma y presente los datos al usuario, desplegado en Vercel.
    
    \item \textbf{Implementar la lógica en el backend} para que sea capaz de:
    \begin{itemize}
        \item Recibir una dirección IP o dominio como entrada mediante endpoint REST POST /api/analysis/analyze.
        \item Resolver dominios a direcciones IP mediante Java InetAddress.
        \item Consultar la base de datos local GeoLite2 (MMDB de 61MB) para información geográfica.
        \item Ejecutar escaneos con Nmap usando TCP connect scan (-sT -sV) para detectar servicios, versiones y puertos abiertos.
        \item Consultar ipapi.co para obtener información actualizada de ASN, ISP y organización.
        \item Consultar la API de NVD para obtener vulnerabilidades (CVEs) asociadas a los servicios detectados mediante CPE matching.
        \item Calcular score de seguridad basado en vulnerabilidades, puertos críticos y servicios desactualizados.
        \item Generar recomendaciones de seguridad mediante sistema experto basado en reglas.
        \item Combinar los resultados de todas las fuentes en una única respuesta estructurada (JSON).
    \end{itemize}
    
    \item \textbf{Desarrollar una interfaz de usuario completa} en el frontend que permita:
    \begin{itemize}
        \item Ingresar una dirección IP o dominio en campo de búsqueda con validación.
        \item Implementar sistema de cache local con LocalForage (IndexedDB) y TTL de 7 días.
        \item Presentar análisis individual con visualización de servicios, vulnerabilidades, mapa geográfico y recomendaciones (Analysis.vue).
        \item Comparar dos análisis lado a lado con resaltado de diferencias (IPComparator.vue).
        \item Visualizar historial de análisis con estadísticas, gráficos Chart.js y gestión de cache (History.vue).
        \item Soportar temas claro/oscuro con persistencia de preferencia.
    \end{itemize}
    
    \item \textbf{Documentar el proyecto}, incluyendo configuración del entorno de desarrollo (requisitos: Java 21, Gradle 9, Node.js 20+), proceso de build con Docker multi-stage, configuración de variables de entorno (CORS\_ORIGINS, VITE\_API\_BASE\_URL, QUARKUS\_PROFILE), guía de despliegue en Railway y Vercel, y documentación de API con ejemplos de uso.
\end{enumerate}
