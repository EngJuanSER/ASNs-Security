\section{Objetivos de la Investigación}

\subsection{Objetivo General}
Diseñar, desarrollar y documentar una herramienta de software funcional que integre datos de geolocalización de la base de datos GeoLite2, escaneo de servicios con Nmap y análisis de vulnerabilidades mediante NVD API para presentar un perfil unificado y de fácil consulta sobre una dirección IP.

\subsection{Objetivos Específicos}

\begin{enumerate}
    \item \textbf{Diseñar una arquitectura de software simple} para la herramienta, basada en un backend (Quarkus) que centralice la lógica de negocio y un frontend (Vue.js) que consuma y presente los datos al usuario.
    
    \item \textbf{Implementar la lógica en el backend} para que sea capaz de:
    \begin{itemize}
        \item Recibir una dirección IP como entrada.
        \item Consultar la base de datos local de GeoLite2 para obtener la información de geolocalización.
        \item Ejecutar escaneos con Nmap para detectar servicios, versiones y puertos abiertos.
        \item Consultar la API de NVD para obtener vulnerabilidades (CVEs) asociadas a los servicios detectados.
        \item Combinar los resultados de todas las fuentes en una única respuesta estructurada (JSON).
    \end{itemize}
    
    \item \textbf{Desarrollar una interfaz de usuario básica} en el frontend que permita:
    \begin{itemize}
        \item Ingresar una dirección IP en un campo de texto.
        \item Enviar la solicitud al backend.
        \item Presentar de forma clara y ordenada la información consolidada recibida del backend.
    \end{itemize}
    
    \item \textbf{Documentar el proyecto}, incluyendo los pasos para la configuración del entorno de desarrollo, la instalación de la herramienta y una guía básica de uso de la API y la interfaz de usuario.
\end{enumerate}
