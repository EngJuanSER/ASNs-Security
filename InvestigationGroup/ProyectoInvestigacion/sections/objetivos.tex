\section{Objetivos de la Investigación}

\subsection{Objetivo General}
Desarrollar e implementar una herramienta de diagnóstico unificado que integre el análisis de seguridad y geolocalización de direcciones IP mediante el uso exclusivo de fuentes abiertas, para fortalecer las capacidades de inteligencia de amenazas cibernéticas en organizaciones colombianas, evaluando su efectividad comparativa con soluciones comerciales existentes.

\subsection{Objetivos Específicos}

\subsubsection{Objetivos de Diseño y Desarrollo}
\begin{enumerate}
    \item \textbf{Diseñar la arquitectura técnica integral} de una plataforma que integre eficientemente los conjuntos de datos de Censys disponibles en Google BigQuery con la base de datos GeoLite2 de MaxMind, considerando aspectos de escalabilidad, rendimiento y mantenibilidad.
    
    \item \textbf{Implementar consultas optimizadas en BigQuery} para la extracción eficiente de información sobre puertos, servicios, certificados SSL/TLS, y metadatos de red asociados a direcciones IP específicas, minimizando costos computacionales y tiempos de respuesta.
    
    \item \textbf{Desarrollar algoritmos de correlación avanzados} que combinen datos de reconocimiento pasivo con información de geolocalización para generar evaluaciones comprehensivas de seguridad, incluyendo scoring de riesgo automatizado y detección de anomalías.
\end{enumerate}

\subsubsection{Objetivos de Interfaz y Usabilidad}
\begin{enumerate}
    \setcounter{enumi}{3}
    \item \textbf{Crear una interfaz de usuario intuitiva y responsiva} que presente los resultados de análisis de forma clara, estructurada y reproducible, incluyendo visualizaciones interactivas, reportes exportables y dashboards personalizables.
    
    \item \textbf{Implementar funcionalidades de análisis masivo} que permitan el procesamiento simultáneo de múltiples direcciones IP con capacidades de filtrado, agrupación y análisis comparativo de resultados.
\end{enumerate}

\subsubsection{Objetivos de Validación y Evaluación}
\begin{enumerate}
    \setcounter{enumi}{5}
    \item \textbf{Validar la efectividad de la herramienta} mediante pruebas comparativas exhaustivas con servicios comerciales existentes (VirusTotal, Shodan, AbuseIPDB) utilizando métricas de precisión, recall, F1-score y tiempo de respuesta.
    
    \item \textbf{Evaluar la aplicabilidad en contextos reales} mediante casos de uso específicos del panorama de ciberseguridad colombiano, incluyendo análisis de campañas de malware locales, investigación de infraestructuras de comando y control, y respuesta a incidentes.
\end{enumerate}

\subsubsection{Objetivos de Transferencia y Sostenibilidad}
\begin{enumerate}
    \setcounter{enumi}{7}
    \item \textbf{Documentar metodologías y mejores prácticas} para la implementación, configuración y uso de la herramienta, facilitando su adopción por organizaciones con diferentes niveles de madurez tecnológica.
    
    \item \textbf{Establecer un marco de evaluación de impacto} que permita medir la contribución de la herramienta al fortalecimiento de capacidades nacionales de ciberseguridad y la reducción de dependencia de soluciones comerciales extranjeras.
\end{enumerate}

\subsection{Objetivos Secundarios}

\subsubsection{Objetivos de Investigación Aplicada}
\begin{itemize}
    \item Caracterizar las limitaciones y ventajas del uso exclusivo de fuentes abiertas en análisis de seguridad IP
    \item Identificar patrones específicos de amenazas en el contexto colombiano mediante análisis de datos históricos
    \item Establecer benchmarks de rendimiento para herramientas de análisis IP basadas en fuentes abiertas
\end{itemize}

\subsubsection{Objetivos de Contribución Académica}
\begin{itemize}
    \item Publicar resultados en conferencias nacionales e internacionales de ciberseguridad
    \item Generar artículos científicos sobre metodologías de integración de fuentes abiertas
    \item Contribuir al desarrollo de estándares nacionales para herramientas de ciberseguridad
\end{itemize}

\subsubsection{Objetivos de Impacto Social}
\begin{itemize}
    \item Democratizar el acceso a herramientas avanzadas de análisis IP para organizaciones con limitaciones presupuestarias
    \item Fortalecer las capacidades técnicas locales mediante transferencia de conocimiento y código abierto
    \item Contribuir a la soberanía digital nacional mediante el desarrollo de capacidades tecnológicas autóctonas
\end{itemize}
