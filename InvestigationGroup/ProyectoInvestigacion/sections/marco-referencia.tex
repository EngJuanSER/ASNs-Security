\section{Marco de Referencia}

\subsection{Marco Teórico}

\subsubsection{Reconocimiento Pasivo en Ciberseguridad}

\textbf{Conceptos Fundamentales:}
El reconocimiento pasivo se define como el proceso de recolección de información sobre sistemas objetivo utilizando fuentes públicas disponibles, sin establecer comunicación directa que pueda ser detectada. Esta técnica presenta ventajas clave para el análisis de seguridad:

\begin{enumerate}
    \item \textbf{No detectabilidad:} No genera tráfico hacia el objetivo
    \item \textbf{Legalidad:} Utiliza información públicamente disponible
    \item \textbf{Escalabilidad:} Permite análisis de múltiples direcciones IP
\end{enumerate}

\textbf{Herramientas y Plataformas Relevantes:}
Las plataformas modernas han democratizado el acceso a datos de reconocimiento masivo:

\begin{itemize}
    \item \textbf{Censys:} Plataforma que proporciona datos de escaneo de Internet mediante BigQuery
    \item \textbf{Shodan:} Motor de búsqueda para dispositivos conectados a Internet
    \item \textbf{VirusTotal:} Agregador de análisis de amenazas y reputación de IP
\end{itemize}

Para este proyecto, nos enfocaremos en los datos de Censys disponibles a través de Google BigQuery, que ofrecen información histórica y actual sobre servicios expuestos en Internet.

\subsubsection{Geolocalización de Direcciones IP}

\textbf{Principios Básicos:}
La geolocalización IP correlaciona direcciones de red con ubicaciones geográficas utilizando diversos métodos:

\begin{itemize}
    \item \textbf{Registros WHOIS:} Información de asignación de bloques IP
    \item \textbf{Mediciones de latencia:} Correlación entre distancia y tiempo de respuesta
    \item \textbf{Bases de datos comerciales:} Agregación de múltiples fuentes de información
\end{itemize}

\textbf{Limitaciones Conocidas:}
\begin{itemize}
    \item Precisión variable según la región geográfica
    \item Influencia de CDNs y tecnologías de distribución de contenido
    \item Diferencias en infraestructura de red entre países
\end{itemize}

\textbf{MaxMind GeoLite2:}
Para este proyecto utilizaremos la base de datos GeoLite2 de MaxMind, que ofrece:
\begin{itemize}
    \item Datos de geolocalización gratuitos actualizados mensualmente
    \item Cobertura global con precisión estimada del 95\% a nivel país
    \item Formato MMDB optimizado para consultas rápidas
    \item API simple para integración en aplicaciones
\end{itemize}

\subsubsection{Inteligencia de Amenazas Cibernéticas}

\textbf{Marcos de Referencia:}
Los frameworks contemporáneos proporcionan estructura para el análisis de amenazas:

\textbf{MITRE ATT\&CK Framework:}
\begin{itemize}
    \item Taxonomía de técnicas de adversarios basada en observaciones reales
    \item 14 tácticas principales desde acceso inicial hasta exfiltración
    \item Base de conocimiento para correlacionar indicadores con técnicas
\end{itemize}

\textbf{Pyramid of Pain:}
\begin{itemize}
    \item Jerarquía de indicadores según la dificultad de evasión para atacantes
    \item Direcciones IP en la base: fáciles de cambiar pero útiles para detección
    \item TTPs en la cima: difíciles de cambiar y más valiosos para defensa
\end{itemize}

\subsection{Marco Conceptual}

\subsubsection{Definiciones Operacionales}

\begin{description}
    \item[Reconocimiento Pasivo:] Recolección de información sobre direcciones IP utilizando fuentes públicas sin contacto directo con los sistemas objetivo.

    \item[Geolocalización IP:] Proceso de determinar la ubicación geográfica aproximada de una dirección IP mediante consultas a bases de datos especializadas.

    \item[Inteligencia de Amenazas:] Información procesada sobre indicadores de compromiso (IOCs) que incluye direcciones IP, dominios y patrones de comportamiento malicioso.

    \item[BigQuery:] Servicio de almacén de datos de Google Cloud que permite consultas SQL sobre grandes volúmenes de datos, incluyendo datasets públicos de Censys.

    \item[Censys:] Plataforma que realiza escaneos regulares de Internet y proporciona datos sobre servicios, certificados y configuraciones accesibles públicamente.

    \item[GeoLite2:] Base de datos gratuita de geolocalización IP proporcionada por MaxMind, actualizada mensualmente.

    \item[Indicator of Compromise (IOC):] Artefacto digital que indica con alta probabilidad una intrusión o actividad maliciosa, incluyendo direcciones IP sospechosas.
\end{description}

\subsubsection{Categorización de Datos IP}

\textbf{Tipos de Información Disponible:}
\begin{itemize}
    \item \textbf{Geográfica:} País, región, ciudad, coordenadas aproximadas
    \item \textbf{Red:} ASN, ISP, tipo de organización
    \item \textbf{Servicios:} Puertos abiertos, protocolos, banners de servicio
    \item \textbf{Reputación:} Historial de actividad maliciosa, categorización de amenazas
\end{itemize}

\subsection{Marco Tecnológico}

\subsubsection{Arquitectura de Datos}

\textbf{Fuentes de Datos Primarias:}
\begin{enumerate}
    \item \textbf{Censys via BigQuery:}
    \begin{itemize}
        \item Datos históricos desde 2015
        \item Actualizaciones semanales
        \item Información de puertos, servicios y certificados
        \item Acceso gratuito con limitaciones de quota
    \end{itemize}
    
    \item \textbf{MaxMind GeoLite2:}
    \begin{itemize}
        \item Base de datos descargable
        \item Actualizaciones mensuales
        \item Formato MMDB para consultas eficientes
        \item Licencia gratuita para uso no comercial
    \end{itemize}
\end{enumerate}

\subsubsection{Stack Tecnológico}

\textbf{Backend - Quarkus:}
\begin{itemize}
    \item Framework Java nativo en la nube
    \item Tiempo de inicio rápido y bajo consumo de memoria
    \item Integración nativa con APIs REST y bases de datos
    \item Soporte para contenedores y despliegue cloud
\end{itemize}

\textbf{Frontend - Vue.js:}
\begin{itemize}
    \item Framework JavaScript progresivo
    \item Curva de aprendizaje suave
    \item Ecosistema rico de componentes
    \item Herramientas de desarrollo integradas
\end{itemize}

\textbf{Infraestructura Cloud:}
\begin{itemize}
    \item Google Cloud Platform para BigQuery
    \item Servicios de hosting para la aplicación web
    \item CDN para distribución de contenido estático
\end{itemize}

\subsection{Marco Espacial y Temporal}

\subsubsection{Contexto del Proyecto}
El proyecto se desarrolla en el contexto académico universitario durante el período octubre-diciembre 2025, con las siguientes características:

\textbf{Alcance Geográfico:}
\begin{itemize}
    \item Enfoque en análisis de IP sin restricción geográfica
    \item Especial atención a datos relevantes para el contexto colombiano
    \item Utilización de fuentes de datos globales
\end{itemize}

\textbf{Limitaciones Temporales:}
\begin{itemize}
    \item Desarrollo incremental en 8 semanas
    \item Datos históricos disponibles desde 2015 (Censys)
    \item Actualizaciones de datos según disponibilidad de fuentes
\end{itemize}

\subsubsection{Consideraciones Técnicas}

\textbf{Limitaciones de Recursos:}
\begin{itemize}
    \item Quotas gratuitas de servicios cloud
    \item Datos limitados a fuentes públicas gratuitas
    \item Capacidad de procesamiento según recursos académicos disponibles
\end{itemize}

\textbf{Restricciones Éticas y Legales:}
\begin{itemize}
    \item Uso exclusivo de datos públicos
    \item Respeto a términos de servicio de proveedores de datos
    \item Implementación de medidas de privacidad por diseño
\end{itemize}


