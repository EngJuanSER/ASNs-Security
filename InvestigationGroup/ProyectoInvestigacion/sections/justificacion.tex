\section{Justificación de la Investigación}

El problema de la fragmentación de información de inteligencia sobre direcciones IP existe porque las fuentes de datos están dispersas en múltiples proveedores, cada uno con diferentes formatos, APIs y costos de acceso. Es importante resolverlo porque los analistas de seguridad en Colombia, especialmente en entornos académicos y organizaciones con recursos limitados, enfrentan barreras significativas para realizar diagnósticos integrales, lo que impacta directamente en la capacidad de respuesta ante amenazas cibernéticas. Esta investigación es necesaria y pertinente porque propone una alternativa práctica basada en fuentes abiertas y gratuitas, democratizando el acceso a herramientas de análisis de seguridad y generando capacidades técnicas locales que contribuyan al fortalecimiento de la ciberseguridad nacional.

\subsection{Justificación Teórica}

\subsubsection{Contribución al Conocimiento Científico}
Esta investigación contribuye al campo de la ciberseguridad mediante el desarrollo de una herramienta práctica que integra fuentes abiertas de datos para el análisis de direcciones IP. Como proyecto académico, busca explorar la viabilidad de combinar datos públicos disponibles (GeoLite2, Nmap, NVD e ipapi.co) para crear una solución funcional de análisis de seguridad.

El trabajo se enfoca en la implementación práctica de técnicas ya establecidas en la literatura~\cite{alghamdi2024reconnaissance, ferrag2019deep}, adaptándolas a un contexto de recursos limitados y fuentes de datos gratuitas. Se busca validar si es posible crear una herramienta útil para análisis integral de IP utilizando únicamente tecnologías y datos de acceso libre.

\subsubsection{Objetivos Académicos}
El proyecto tiene como objetivo académico principal desarrollar competencias prácticas en:

\begin{itemize}
    \item \textbf{Integración de Datos:} Uso de APIs y bases de datos públicas para ciberseguridad
    \item \textbf{Desarrollo Web:} Implementación de aplicaciones usando tecnologías modernas
    \item \textbf{Análisis de Seguridad:} Aplicación práctica de conceptos de reconocimiento pasivo
    \item \textbf{Investigación Aplicada:} Documentación y evaluación sistemática de resultados
\end{itemize}

\subsection{Justificación Metodológica}

\subsubsection{Enfoque Práctico}
La metodología se basa en el desarrollo incremental de una aplicación web que integre datos de fuentes públicas confiables. Se utilizará un stack tecnológico moderno pero accesible:

\textbf{Fuentes de Datos:}
\begin{itemize}
    \item \textbf{MaxMind GeoLite2:} Base de datos gratuita de geolocalización IP, validada en estudios académicos~\cite{schopman2021validating}
    \item \textbf{Nmap:} Herramienta de código abierto para escaneo de red y detección de servicios~\cite{lyon2009nmap, pittman2023comparative}
    \item \textbf{NVD API:} API pública de NIST para consulta de vulnerabilidades (CVEs)
    \item \textbf{ipapi.co:} API REST gratuita para obtención de información de ASN e ISP
\end{itemize}

\textbf{Stack Tecnológico:}
\begin{itemize}
    \item \textbf{Backend:} Quarkus (Java) para APIs y lógica de negocio
    \item \textbf{Frontend:} Vue.js para interfaz de usuario
    \item \textbf{Escaneo:} Nmap integrado en backend para detección de servicios
    \item \textbf{Infraestructura:} Servicios cloud básicos para despliegue
\end{itemize}

\subsubsection{Validación Práctica}
La validación se realizará mediante:
\begin{itemize}
    \item Pruebas funcionales de la aplicación desarrollada
    \item Comparación básica con herramientas existentes
    \item Evaluación de usabilidad con usuarios del entorno académico
    \item Documentación de lecciones aprendidas y limitaciones encontradas
\end{itemize}

\subsection{Justificación Práctica}

\subsubsection{Aplicabilidad en el Entorno Académico}
La herramienta desarrollada tendrá aplicación directa en:

\textbf{Educación en Ciberseguridad:}
\begin{itemize}
    \item Material práctico para cursos de seguridad informática
    \item Ejemplo de integración de datos para análisis de amenazas
    \item Caso de estudio para desarrollo de aplicaciones de seguridad
\end{itemize}

\textbf{Investigación Estudiantil:}
\begin{itemize}
    \item Base para futuros proyectos de grado relacionados
    \item Herramienta para análisis básico en investigaciones de seguridad
    \item Ejemplo de implementación con recursos limitados
\end{itemize}

\subsubsection{Contribución al Programa Académico}
El proyecto contribuye al programa de estudios mediante:

\begin{itemize}
    \item \textbf{Aplicación Práctica:} Implementación real de conceptos teóricos estudiados
    \item \textbf{Tecnologías Actuales:} Experiencia con herramientas y frameworks modernos
    \item \textbf{Metodología de Desarrollo:} Aplicación de buenas prácticas de ingeniería de software
    \item \textbf{Documentación Técnica:} Desarrollo de habilidades de comunicación técnica
\end{itemize}

\subsubsection{Viabilidad y Alcance Realista}
El proyecto está diseñado para ser completado exitosamente en el tiempo disponible:

\textbf{Recursos Disponibles:}
\begin{itemize}
    \item Acceso a datasets públicos sin costo
    \item Herramientas de código abierto (Nmap, NVD API)
    \item Herramientas de desarrollo gratuitas o con licencias estudiantiles
    \item Supervisión académica y acceso a recursos universitarios
\end{itemize}

\textbf{Limitaciones Reconocidas:}
\begin{itemize}
    \item Alcance limitado a fuentes de datos gratuitas
    \item Funcionalidades básicas en la versión inicial
    \item Evaluación limitada al entorno académico
    \item No pretende competir con soluciones comerciales especializadas
\end{itemize}

\subsubsection{Impacto Esperado}
Se espera que el proyecto genere:

\begin{itemize}
    \item Una herramienta funcional para análisis básico de IP
    \item Documentación técnica que sirva como referencia para futuros estudiantes
    \item Experiencia práctica en desarrollo de aplicaciones de ciberseguridad
    \item Código fuente abierto que pueda ser mejorado por la comunidad académica
\end{itemize}

El proyecto se enfoca en demostrar que es posible crear herramientas útiles de ciberseguridad usando recursos disponibles públicamente, proporcionando una alternativa educativa a las costosas soluciones comerciales.
