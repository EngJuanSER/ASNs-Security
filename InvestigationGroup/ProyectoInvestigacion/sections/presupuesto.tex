\section{Presupuesto}

\subsection{Contexto del Presupuesto}

Este presupuesto está diseñado para un proyecto académico desarrollado por dos estudiantes universitarios durante un período de 8 semanas. Se enfoca en recursos realmente necesarios y aprovecha herramientas gratuitas y créditos académicos disponibles.

\subsection{Filosofía de Costos}

El proyecto se basa en el principio de \textbf{desarrollo económico} utilizando:
\begin{itemize}
    \item Herramientas y servicios gratuitos o con licencias estudiantiles
    \item Créditos académicos para servicios cloud
    \item Hardware y recursos ya disponibles
    \item Fuentes de datos públicas sin costo
\end{itemize}

\subsection{Estructura de Costos}

\subsubsection{Servicios Cloud y Datos}

\begin{table}[H]
    \centering
    \small
    \begin{tabular}{|l|l|r|}
        \hline
        \textbf{Servicio} & \textbf{Descripción} & \textbf{Costo} \\
        \hline
        Google Cloud Platform & Créditos académicos (\$300 USD) & \$0 \\
        \hline
        BigQuery (Censys data) & Consultas dentro de quota gratuita (1TB/mes) & \$0 \\
        \hline
        Compute Engine & Instancia básica para desarrollo (2 meses) & \$0 \\
        \hline
        Cloud Storage & Almacenamiento de datos y backups & \$0 \\
        \hline
        \textbf{Subtotal Cloud} & & \textbf{\$0} \\
        \hline
    \end{tabular}
    \caption{Costos de Servicios Cloud}
    \label{tab:costos_cloud}
\end{table}

\subsubsection{Software y Herramientas}

\begin{table}[H]
    \centering
    \small
    \begin{tabular}{|l|l|r|}
        \hline
        \textbf{Herramienta} & \textbf{Tipo de Licencia} & \textbf{Costo} \\
        \hline
        IntelliJ IDEA Community & Gratuita & \$0 \\
        \hline
        Visual Studio Code & Gratuita & \$0 \\
        \hline
        GitHub & Cuenta estudiante gratuita & \$0 \\
        \hline
        Docker Desktop & Uso personal gratuito & \$0 \\
        \hline
        Postman & Plan gratuito & \$0 \\
        \hline
        MaxMind GeoLite2 & Licencia gratuita no comercial & \$0 \\
        \hline
        \textbf{Subtotal Software} & & \textbf{\$0} \\
        \hline
    \end{tabular}
    \caption{Licencias de Software}
    \label{tab:software}
\end{table}

\subsubsection{Recursos de Desarrollo}

\begin{table}[H]
    \centering
    \small
    \begin{tabular}{|l|l|r|}
        \hline
        \textbf{Recurso} & \textbf{Descripción} & \textbf{Costo} \\
        \hline
        Computadoras personales & Equipos existentes de los estudiantes & \$0 \\
        \hline
        Conexión a Internet & Internet doméstico/universitario & \$0 \\
        \hline
        Espacio físico de trabajo & Aulas y espacios universitarios & \$0 \\
        \hline
        \textbf{Subtotal Recursos} & & \textbf{\$0} \\
        \hline
    \end{tabular}
    \caption{Recursos de Desarrollo}
    \label{tab:recursos_desarrollo}
\end{table}

\subsubsection{Costos Opcionales (Si se requieren)}

\begin{table}[H]
    \centering
    \small
    \begin{tabular}{|l|l|r|}
        \hline
        \textbf{Concepto} & \textbf{Descripción} & \textbf{Costo Estimado} \\
        \hline
        Dominio personalizado & .com o .net para demo (1 año) & \$15.000 \\
        \hline
        Hosting adicional & Si se requiere más allá de GCP gratuito & \$30.000 \\
        \hline
        Certificado SSL & Si no está incluido en hosting & \$0 \\
        \hline
        Servicios de validación & APIs premium para comparación & \$50.000 \\
        \hline
        \textbf{Subtotal Opcional} & & \textbf{\$95.000} \\
        \hline
    \end{tabular}
    \caption{Costos Opcionales}
    \label{tab:costos_opcionales}
\end{table}

\subsection{Tiempo y Dedicación}

\subsubsection{Estimación de Horas}

\begin{table}[H]
    \centering
    \small
    \begin{tabular}{|l|c|c|c|}
        \hline
        \textbf{Actividad} & \textbf{Horas/Semana} & \textbf{Total Semanas} & \textbf{Horas Total} \\
        \hline
        Desarrollo Backend & 6 & 4 & 24 \\
        \hline
        Desarrollo Frontend & 6 & 4 & 24 \\
        \hline
        Testing y Validación & 4 & 2 & 8 \\
        \hline
        Documentación & 3 & 3 & 9 \\
        \hline
        \textbf{Total por Estudiante} & & & \textbf{65 horas} \\
        \hline
        \textbf{Total Proyecto (2 estudiantes)} & & & \textbf{130 horas} \\
        \hline
    \end{tabular}
    \caption{Distribución de Tiempo por Estudiante}
    \label{tab:tiempo_estudiantes}
\end{table}

\subsection{Análisis de Viabilidad Económica}

\subsubsection{Ventajas del Modelo Económico}

\textbf{Costo-Beneficio Favorable:}
\begin{itemize}
    \item Inversión monetaria mínima (\$100.000 COP total)
    \item Aprovechamiento máximo de recursos académicos
    \item Experiencia práctica con herramientas profesionales
    \item Desarrollo de portafolio técnico para estudiantes
\end{itemize}

\textbf{Sostenibilidad:}
\begin{itemize}
    \item No dependencia de presupuestos institucionales grandes
    \item Escalable a proyectos futuros similares
    \item Replicable por otros estudiantes
    \item Base para proyectos de grado posteriores
\end{itemize}

\subsubsection{Comparación con Alternativas Comerciales}

\begin{table}[H]
    \centering
    \small
    \begin{tabular}{|l|r|r|}
        \hline
        \textbf{Opción} & \textbf{Costo Mensual} & \textbf{Costo 2 Meses} \\
        \hline
        Nuestro proyecto & \$0 -- \$50.000 & \$0 -- \$100.000 \\
        \hline
        VirusTotal Enterprise & \$2.000.000 & \$4.000.000 \\
        \hline
        Shodan Enterprise & \$1.500.000 & \$3.000.000 \\
        \hline
        MaxMind GeoIP2 Precision & \$800.000 & \$1.600.000 \\
        \hline
        \textbf{Ahorro estimado} & & \textbf{\$8.500.000} \\
        \hline
    \end{tabular}
    \caption{Comparación de Costos}
    \label{tab:comparacion_costos}
\end{table}

\subsection{Gestión de Riesgos Financieros}

\subsubsection{Riesgos Identificados}

\begin{table}[H]
    \centering
    \small
    \begin{tabular}{|l|c|l|}
        \hline
        \textbf{Riesgo} & \textbf{Probabilidad} & \textbf{Mitigación} \\
        \hline
        Agotamiento de créditos GCP & Baja & Monitoreo de uso, optimización \\
        \hline
        Necesidad de recursos adicionales & Media & Fuentes alternativas gratuitas \\
        \hline
        Costos no previstos & Baja & Buffer de \$100.000 COP \\
        \hline
    \end{tabular}
    \caption{Riesgos Financieros}
    \label{tab:riesgos_financieros}
\end{table}

\subsection{Retorno de Inversión Académico}

\subsubsection{Beneficios Intangibles}

\textbf{Para los Estudiantes:}
\begin{itemize}
    \item Experiencia práctica con tecnologías modernas
    \item Portfolio de desarrollo para empleabilidad
    \item Comprensión de arquitecturas cloud reales
    \item Metodologías de desarrollo ágil
\end{itemize}

\textbf{Para la Institución:}
\begin{itemize}
    \item Proyecto de referencia para futuros estudiantes
    \item Demostración de capacidades del programa
    \item Contribución al repositorio académico
    \item Posible base para investigaciones futuras
\end{itemize}

\subsubsection{Valor Estimado del Proyecto}

\begin{table}[H]
    \centering
    \small
    \begin{tabular}{|l|r|}
        \hline
        \textbf{Concepto} & \textbf{Valor Estimado} \\
        \hline
        Tiempo de desarrollo (130 horas @ \$30.000/hora) & \$3.900.000 \\
        \hline
        Valor educativo (experiencia práctica) & \$2.000.000 \\
        \hline
        Código fuente reutilizable & \$1.000.000 \\
        \hline
        Documentación técnica & \$500.000 \\
        \hline
        \textbf{Valor Total Generado} & \textbf{\$7.400.000} \\
        \hline
        \textbf{Inversión Real} & \textbf{\$100.000} \\
        \hline
        \textbf{ROI Académico} & \textbf{7.300\%} \\
        \hline
    \end{tabular}
    \caption{Valor Estimado vs Inversión}
    \label{tab:valor_proyecto}
\end{table}

\subsection{Conclusiones del Presupuesto}

\subsubsection{Viabilidad Financiera}
El proyecto es altamente viable desde el punto de vista económico, requiriendo una inversión mínima o nula mientras genera valor significativo en términos de:

\begin{itemize}
    \item Aprendizaje práctico y aplicado
    \item Desarrollo de competencias técnicas
    \item Creación de herramientas útiles para la comunidad académica
    \item Base para futuros proyectos de investigación
\end{itemize}

\subsubsection{Sostenibilidad del Modelo}
Este modelo presupuestario demuestra que es posible desarrollar proyectos técnicos valiosos con recursos académicos existentes, estableciendo un precedente para futuras iniciativas estudiantiles similares.
