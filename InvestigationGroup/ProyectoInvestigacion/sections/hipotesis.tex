\section{Hipótesis de la Investigación}

\subsection{Hipótesis Principal}

Es factible desarrollar una herramienta de software que, utilizando exclusivamente las fuentes de datos gratuitas de GeoLite2 y Censys (a través de la cuota gratuita de BigQuery), puede proporcionar un perfil de información consolidado sobre una dirección IP que es útil y suficiente para un análisis de seguridad inicial.

\subsection{Sub-hipótesis}

\begin{enumerate}
    \item \textbf{Viabilidad Técnica:} La integración de una base de datos local (GeoLite2) y una API remota (BigQuery para Censys) en una única aplicación (backend con Quarkus, frontend con Vue.js) es técnicamente realizable dentro del marco de un proyecto estudiantil de corta duración.
    
    \item \textbf{Suficiencia de Datos:} La combinación de datos de geolocalización (país, ciudad) y datos de infraestructura (puertos abiertos, servicios) es suficiente para que un analista o desarrollador pueda tomar una decisión informada preliminar sobre la naturaleza de una dirección IP.
    
    \item \textbf{Eficiencia del Proceso:} Una herramienta que automatiza la consulta y consolidación de estas dos fuentes de datos reducirá significativamente el tiempo y el esfuerzo manual en comparación con la consulta separada de cada fuente.
\end{enumerate}

\subsection{Variables}

\begin{itemize}
    \item \textbf{Variable Independiente:} La herramienta de software desarrollada que integra GeoLite2 y Censys.
    \item \textbf{Variable Dependiente:} La utilidad y eficiencia del perfil de IP generado, medida en términos de:
    \begin{itemize}
        \item \textbf{Completitud de la información:} ¿Contiene los datos esperados de ambas fuentes?
        \item \textbf{Tiempo de respuesta:} ¿Cuánto tarda en generar un informe?
        \item \textbf{Facilidad de uso:} ¿La información se presenta de manera clara y comprensible?
    \end{itemize}
\end{itemize}
