\usepackage[T1]{fontenc}
\usepackage[utf8]{inputenc}
\usepackage[spanish]{babel}
\usepackage{lmodern}
\usepackage{microtype}

\usepackage{amsmath,amssymb,amsfonts}
\usepackage{graphicx}
\usepackage{textcomp}
\usepackage{xcolor}
\usepackage{hyperref}
\usepackage{geometry}
\usepackage{fancyhdr}
\usepackage{setspace}
\usepackage{cite}

% Paquetes para tablas y posicionamiento (según documentación oficial)
\usepackage{float}           % Para [H] - posición exacta
\usepackage{placeins}        % Para \FloatBarrier
\usepackage{array}           % Mejoras en tabular
\usepackage{tabularx}        % Tablas con ancho variable
\usepackage{booktabs}        % Líneas profesionales
\usepackage{multirow}        % Celdas multi-fila
\usepackage{colortbl}        % Colores en tablas
\usepackage[position=bottom]{caption} % Control de captions y posición debajo

% Configuración de página
\geometry{left=3cm,right=3cm,top=3cm,bottom=3cm}
\onehalfspacing

% Configuración de flotantes (según documentación oficial LaTeX)
\setcounter{topnumber}{3}          % Máx floats arriba: 3
\setcounter{bottomnumber}{2}       % Máx floats abajo: 2  
\setcounter{totalnumber}{4}        % Máx floats por página: 4
\renewcommand{\topfraction}{0.8}    % 80% página para floats arriba
\renewcommand{\bottomfraction}{0.5} % 50% página para floats abajo
\renewcommand{\textfraction}{0.2}   % Mín 20% texto por página
\renewcommand{\floatpagefraction}{0.8} % 80% lleno para página float

% Configuración de encabezados
\pagestyle{fancy}
\fancyhf{}
\fancyhead[L]{Proyecto de Investigación - Herramienta de Diagnóstico IP}
\fancyhead[R]{\thepage}
\setlength{\headheight}{15pt}
\renewcommand{\headrulewidth}{0.4pt}

% Configuración de hyperref
\hypersetup{
    colorlinks=true,
    linkcolor=blue,
    filecolor=magenta,
    urlcolor=cyan,
    citecolor=green
}

% Configuración de tablas
\setlength{\tabcolsep}{6pt}         % Espacio entre columnas
\renewcommand{\arraystretch}{1.2}    % Altura de filas

% Comandos personalizados
\newcommand{\todo}[1]{\textcolor{red}{\textbf{TODO: #1}}}
\newcommand{\note}[1]{\textcolor{blue}{\textbf{Nota: #1}}}
