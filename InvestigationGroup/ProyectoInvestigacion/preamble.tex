\usepackage[utf8]{inputenc}
\usepackage[spanish]{babel}
\usepackage{amsmath,amssymb,amsfonts}
\usepackage{graphicx}
\usepackage{textcomp}
\usepackage{xcolor}
\usepackage{hyperref}
\usepackage{float}
\usepackage{geometry}
\usepackage{fancyhdr}
\usepackage{setspace}
\usepackage{cite}

% Paquetes esenciales para tablas
\usepackage{array}
\usepackage{tabularx}
\usepackage{booktabs}
\usepackage{caption}
\usepackage{multirow}
\usepackage{colortbl}
\usepackage{needspace} % evita que queden títulos sin contenido en la misma página

% Helper: asegura que quepan título + tabla al inicio de una subsección
\newcommand{\SubsectionNeedsSpace}{\Needspace{16\baselineskip}}

% Configuración de página
\geometry{left=3cm,right=3cm,top=3cm,bottom=3cm}
\onehalfspacing

% Ajustes para minimizar problemas con tablas
\setlength{\emergencystretch}{3em}

% Configuración de encabezados
\pagestyle{fancy}
\fancyhf{}
\fancyhead[L]{Proyecto de Investigación - Herramienta de Diagnóstico IP}
\fancyhead[R]{\thepage}
\setlength{\headheight}{15pt}
\renewcommand{\headrulewidth}{0.4pt}

% Configuración de hyperref
\hypersetup{
    colorlinks=true,
    linkcolor=blue,
    filecolor=magenta,
    urlcolor=cyan,
    citecolor=green
}

% Configuración de colores para tablas
\definecolor{lightgray}{gray}{0.9}

% Configuración de tablas
\setlength{\tabcolsep}{5pt}
\renewcommand{\arraystretch}{1.1}

% Comandos personalizados
\newcommand{\todo}[1]{\textcolor{red}{\textbf{TODO: #1}}}
\newcommand{\note}[1]{\textcolor{blue}{\textbf{Nota: #1}}}
