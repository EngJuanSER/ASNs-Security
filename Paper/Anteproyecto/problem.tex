\section{FORMULACIÓN DEL PROBLEMA}

La infraestructura de internet de Bogotá es un ecosistema complejo, sostenido por múltiples Sistemas Autónomos (ASNs) pertenecientes a proveedores de servicios (ISPs), universidades y grandes corporaciones. La configuración y el mantenimiento de los miles de dispositivos dentro de estas redes determinan su "higiene digital" colectiva. La exposición innecesaria de servicios —como bases de datos, paneles de acceso remoto o protocolos de comunicación obsoletos— crea una superficie de ataque que puede ser explotada por actores maliciosos, resultando en brechas de datos y disrupción de servicios que afectan a miles de ciudadanos y empresas.

Actualmente, no existe un benchmark o un estudio comparativo público, basado en datos empíricos a gran escala, que evalúe la postura de seguridad de estos ASNs clave para Bogotá. Se desconoce qué redes presentan una mayor concentración de configuraciones riesgosas y cuáles son los patrones de exposición más comunes. Sin esta información, los esfuerzos para mejorar la seguridad digital de la ciudad carecen de un diagnóstico claro y priorizado.

Este proyecto propone llenar ese vacío mediante el análisis de los conjuntos de datos públicos de escaneo de internet de Censys. En lugar de depender de APIs restringidas, se utilizará la plataforma Google BigQuery para consultar estos datos masivos, partiendo de una lista autoritaria de ASNs relevantes para Bogotá identificados a través de LACNIC. El objetivo es crear un informe cuantitativo y reproducible de la exposición digital de la infraestructura de red de la ciudad.