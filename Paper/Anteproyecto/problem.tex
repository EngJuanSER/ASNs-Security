\section{FORMULACIÓN DEL PROBLEMA}

Tanto usuarios técnicos como ciudadanos comunes se enfrentan a una creciente necesidad de evaluar la legitimidad y seguridad de las direcciones IP con las que interactúan. Ya sea al analizar los registros de un firewall, verificar la procedencia de un correo electrónico o investigar un servicio en línea, la información requerida para un diagnóstico completo se encuentra dispersa en múltiples plataformas: una para consultar puertos abiertos, otra para la geolocalización, una tercera para la reputación y una cuarta para identificar al propietario de la red.

Esta fragmentación de datos representa una barrera significativa para la toma de decisiones informadas en materia de ciberseguridad. La falta de una herramienta unificada, accesible y de código abierto obliga a realizar un proceso manual, lento e ineficiente, que a menudo queda fuera del alcance de usuarios no especializados.

Este proyecto propone solucionar este problema mediante el desarrollo de una herramienta web de diagnóstico centralizada. Dicha herramienta permitirá a un usuario ingresar una dirección IP o dominio y recibir un informe consolidado que integre datos de seguridad pasiva, geolocalización aproximada, reputación y contexto de red. El objetivo es crear una base sólida para una plataforma de reconocimiento pasivo que democratice el acceso a la inteligencia de amenazas, utilizando exclusivamente herramientas y fuentes de datos gratuitas.