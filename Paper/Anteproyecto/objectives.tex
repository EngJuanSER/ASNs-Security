\section{OBJETIVOS ESPECÍFICOS}
\subsection{Objetivo 1: Identificación y Caracterización de la Exposición por ASN}

Realizar un análisis de dos fases para identificar y cuantificar los servicios expuestos en las redes más importantes de Bogotá:

\textbf{Identificación de ASNs:} Utilizar la base de datos pública de LACNIC (Registro de Direcciones de Internet de América Latina y el Caribe) para identificar y documentar los números de Sistema Autónomo (ASN) de los principales ISPs y entidades académicas que operan en Bogotá.

\textbf{Caracterización de Servicios:} Ejecutar consultas SQL sobre los conjuntos de datos públicos de Censys en Google BigQuery para analizar los ASNs identificados, cuantificando la prevalencia de:
\begin{itemize}
  \item Protocolos de texto plano y obsoletos (Telnet, FTP).
  \item Servicios de bases de datos expuestas (MongoDB, Redis, MySQL, Elasticsearch).
  \item Paneles de control de acceso remoto (RDP, VNC).
  \item Dispositivos de infraestructura con interfaces de gestión públicas (ej. routers con HTTP/HTTPS).
\end{itemize}

\subsection{Objetivo 2: Desarrollo de un Modelo Cuantitativo de Puntuación de Higiene Digital}

Diseñar un modelo de puntuación que permita evaluar y comparar la postura de seguridad de cada ASN de forma objetiva. El modelo:

\begin{itemize}
  \item Asignará un factor de riesgo a cada tipo de servicio expuesto, considerando la probabilidad de explotación y el impacto potencial (ej. un RDP expuesto tiene un riesgo mayor que un FTP).
  \item Normalizará los resultados en función del tamaño del espacio de direcciones IP del ASN para permitir una comparación justa entre redes de diferente escala.
  \item Calculará una "Puntuación de Higiene Digital" final para cada ASN, clasificándolos para facilitar la identificación de aquellos con mayor área de mejora.
\end{itemize}

\subsection{Objetivo 3: Implementación de un Dashboard de Inteligencia de Amenazas Públicas}

Desarrollar una aplicación web (dashboard) para presentar los resultados de forma clara e interactiva. La plataforma no requerirá inicio de sesión y visualizará los datos agregados:

\begin{itemize}
  \item Mostrará un ranking de los ASNs analizados según su Puntuación de Higiene Digital.
  \item Presentará gráficos estadísticos sobre los servicios inseguros más comunes encontrados en toda la infraestructura de Bogotá.
  \item Permitirá filtrar y explorar los resultados por ASN específico, para ver en detalle su perfil de exposición.
  \item Incluirá una sección educativa que explique los riesgos asociados a cada servicio expuesto y ofrezca recomendaciones generales para mitigarlos.
\end{itemize}
