\section{OBJETIVOS ESPECÍFICOS}

\subsection{Objetivo 1: Integración de Fuentes de Datos de Reconocimiento Pasivo Gratuitas}
Desarrollar un sistema backend capaz de consultar y unificar información de diversas fuentes de datos 100\% gratuitas, garantizando la viabilidad del proyecto sin incurrir en costos. Las fuentes a integrar son:

\begin{itemize}
    \item \textbf{Análisis de Puertos y Servicios:} Ejecutar consultas SQL sobre los conjuntos de datos públicos de Censys en Google BigQuery para obtener un perfil histórico de los servicios y puertos expuestos por la IP objetivo, operando dentro del nivel gratuito de la plataforma.
    \item \textbf{Geolocalización Aproximada Local:} Descargar e implementar la base de datos GeoLite2 City de MaxMind. Las consultas de geolocalización se realizarán localmente contra este archivo, lo que garantiza un rendimiento extremadamente rápido y elimina por completo los límites de uso.
    \item \textbf{Contexto de Red y Propiedad:} Realizar consultas WHOIS a los servidores públicos de los Registros Regionales de Internet (como LACNIC para la región) para identificar el Número de Sistema Autónomo (ASN), el ISP propietario del bloque de IP y los datos de contacto de abuso.
    \item \textbf{Reputación y Amenazas:} Integrar la API gratuita de AbuseIPDB para verificar si la dirección IP ha sido reportada por actividades maliciosas y obtener un puntaje de confianza, gestionando el uso dentro de los límites establecidos por su plan gratuito.
\end{itemize}

\subsection{Objetivo 2: Desarrollo de un Modelo de Puntuación de Riesgo y Contexto}
Diseñar un modelo algorítmico que procese los datos recolectados para generar una evaluación clara y concisa de la IP analizada. El modelo deberá:

\begin{itemize}
    \item Asignar un factor de riesgo a los servicios expuestos, priorizando aquellos de mayor criticidad (ej. RDP, SMB, bases de datos abiertas).
    \item Incorporar el puntaje de reputación de AbuseIPDB como un factor clave en la evaluación de riesgo general.
    \item Presentar la información de geolocalización y propiedad de forma clara, enfatizando siempre la naturaleza aproximada de la ubicación.
    \item Calcular una puntuación de riesgo final (ej. Bajo, Medio, Alto, Malicioso Conocido) para la IP analizada.
\end{itemize}

\subsection{Objetivo 3: Implementación de un Dashboard de Diagnóstico Interactivo}
Desarrollar una aplicación web de cara al usuario que sirva como la interfaz para la herramienta. El dashboard deberá:

\begin{itemize}
    \item Proveer una interfaz simple donde el usuario pueda introducir una dirección IP o un dominio para su análisis.
    \item Presentar el informe consolidado en una vista organizada y fácil de entender, con secciones para seguridad, ubicación, reputación y propiedad.
    \item Visualizar la ubicación aproximada (ciudad/país) obtenida de la base de datos GeoLite2 en un mapa interactivo.
    \item Ofrecer recomendaciones de seguridad contextuales y automatizadas basadas en los hallazgos (ej. "Puerto 3389 (RDP) abierto. Este es un riesgo alto de ataque de ransomware").
\end{itemize}
