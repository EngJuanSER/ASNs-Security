\section{RESULTADOS ESPERADOS}

\subsection{Resultado Principal: Observatorio Público de la Higiene Digital de Redes en Bogotá}

Una plataforma web analítica y estática (actualizada periódicamente) que sirva como una fuente de información pública, presentando:

\begin{itemize}
  \item \textbf{Ranking de ASNs:} Una tabla comparativa que clasifica los principales proveedores y redes de Bogotá por su postura de seguridad, basada en datos cuantificables.
  \item \textbf{Perfil de Riesgo de la Ciudad:} Estadísticas agregadas que respondan a preguntas como: ¿Cuál es el protocolo inseguro más común en Bogotá? ¿Cuántas bases de datos están potencialmente expuestas?
  \item \textbf{Visualización de Datos:} Gráficos interactivos que muestren la distribución geográfica de los ASNs con riesgos críticos.
\end{itemize}

\subsection{Resultado Secundario: Manual Metodológico para Análisis de Exposición Digital}

Un documento técnico que detalla el proceso de análisis, incluyendo:

\begin{itemize}
  \item Descripción del método de recolección de datos (Censys, LACNIC, Google BigQuery).
  \item Procedimiento para la asignación de puntuaciones de riesgo.
  \item Herramientas y software utilizados (Python, Pandas, Matplotlib, Flask).
  \item Recomendaciones para la implementación de medidas de seguridad en redes de Bogotá.
\end{itemize}

\begin{itemize}
  \item Descripción del método de recolección de datos (Censys, LACNIC, Google BigQuery).
  \item Procedimiento para la asignación de puntuaciones de riesgo.
  \item Herramientas y software utilizados (Python, Pandas, Matplotlib, Flask).
  \item Recomendaciones para la implementación de medidas de seguridad en redes de Bogotá.
\end{itemize}
