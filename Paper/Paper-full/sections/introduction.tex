\section{Introducción}
La creciente sofisticación de las amenazas cibernéticas y la necesidad de evaluaciones de seguridad precisas han convertido el análisis de direcciones IP en una actividad fundamental para profesionales de ciberseguridad, administradores de sistemas y usuarios técnicos [1]. En el contexto actual, donde los ataques cibernéticos se han incrementado exponencialmente y las organizaciones enfrentan desafíos sin precedentes para proteger su infraestructura digital, la capacidad de evaluar rápida y efectivamente la legitimidad y el nivel de riesgo asociado a una dirección IP específica se ha vuelto crítica.

El reconocimiento pasivo, definido como el proceso de recolección de información sin interactuar directamente con los sistemas objetivo, representa una metodología fundamental en el arsenal de herramientas de ciberseguridad [3]. A diferencia del reconocimiento activo, que implica el envío de consultas directas y puede ser detectado por sistemas de seguridad, el reconocimiento pasivo permite obtener información valiosa manteniendo un perfil bajo y minimizando el riesgo de detección. Esta aproximación es particularmente relevante en evaluaciones de seguridad donde la discreción es fundamental y en contextos donde se requiere un análisis preliminar antes de proceder con técnicas más invasivas.

La geolocalización de direcciones IP, aunque inherentemente aproximada debido a las limitaciones técnicas de los métodos disponibles, proporciona información contextual valiosa para el análisis de seguridad [4]. Los avances recientes en técnicas de geolocalización han demostrado mejoras significativas en la precisión, particularmente cuando se combinan múltiples fuentes de datos y se implementan algoritmos de machine learning para procesar la información recolectada. Sin embargo, la fragmentación de la información en múltiples plataformas y bases de datos presenta desafíos significativos para los analistas de seguridad.

La inteligencia de amenazas cibernéticas ha evolucionado considerablemente en los últimos años, transitioning from manual analysis to sophisticated automated systems powered by artificial intelligence [5]. Esta evolución ha sido impulsada por la necesidad de procesar volúmenes masivos de datos en tiempo real y la complejidad creciente de los patrones de ataque. Las plataformas modernas de inteligencia de amenazas integran múltiples fuentes de datos, utilizan algoritmos avanzados de análisis y proporcionan capacidades predictivas que permiten a las organizaciones adoptar una postura proactiva frente a las amenazas emergentes [2].

En Colombia, como en muchos países en desarrollo, existe una brecha significativa en el acceso a herramientas avanzadas de análisis de seguridad cibernética. Esta situación se ve agravada por los costos asociados a plataformas comerciales especializadas y la falta de soluciones adaptadas al contexto local. La democratización del acceso a estas capacidades representa una oportunidad fundamental para fortalecer la postura de ciberseguridad del país y contribuir al desarrollo de una comunidad técnica más preparada para enfrentar los desafíos actuales.

El problema central que aborda esta investigación es la fragmentación de la información necesaria para realizar un diagnóstico completo de seguridad de direcciones IP. Actualmente, los analistas deben consultar múltiples plataformas: una para verificar puertos abiertos, otra para geolocalización, una tercera para información de reputación, y una cuarta para identificar el propietario de la red. Este proceso manual no solo es ineficiente en términos de tiempo, sino que también aumenta la probabilidad de errores y dificulta la correlación efectiva de información de múltiples fuentes.

La presente investigación propone el desarrollo de una herramienta web unificada que integre estas funcionalidades dispersas en una plataforma coherente y accesible. La solución propuesta se basa en la utilización exclusiva de fuentes de datos gratuitas, garantizando la sostenibilidad y replicabilidad del proyecto. La implementación incluye la integración de datos de Censys a través de Google BigQuery, la base de datos GeoLite2 de MaxMind para geolocalización, consultas WHOIS para información de contexto de red, y la API de AbuseIPDB para datos de reputación.

La contribución principal de este trabajo radica en la creación de una plataforma que no solo consolida información dispersa, sino que también implementa un modelo de puntuación de riesgo que procesa y analiza los datos recolectados para proporcionar evaluaciones comprensivas y contextualizadas. Esta aproximación va más allá de la simple agregación de datos, incorporando lógica de análisis que facilita la toma de decisiones informadas por parte de los usuarios.
