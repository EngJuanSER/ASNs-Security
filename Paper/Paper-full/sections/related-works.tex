\section{Trabajos Relacionados}

La literatura reciente en el área de diagnóstico unificado de seguridad IP, geolocalización y plataformas de inteligencia de amenazas presenta diversas aproximaciones metodológicas que han contribuido significativamente al avance del campo. Esta sección examina diez trabajos relevantes publicados entre 2022-2025, analizando sus metodologías y estableciendo comparaciones con la propuesta de este proyecto.

\subsection{Geolocalización IP y Técnicas de Análisis de Red}

Darwich et al.~(2023)~\cite{6} presentan una metodología replicable para crear conjuntos de datos de geolocalización IP a escala de Internet mediante la combinación de técnicas de medición activa y pasiva. Su aproximación integra datos de traceroute, mediciones de latencia y bases de datos públicas para mejorar la precisión de la geolocalización. La metodología emplea algoritmos de triangulación geométrica y análisis de topología de red para inferir ubicaciones aproximadas. Esta investigación proporciona fundamentos metodológicos sólidos para la geolocalización, aunque se enfoca exclusivamente en la precisión geográfica sin integrar aspectos de seguridad.

Wang et al.~(2025)~\cite{7} introducen NeighborGeo, un modelo novel basado en aprendizaje de estructura de grafos para geolocalización IP. Su metodología utiliza redes neuronales gráficas (GNN) para modelar las relaciones entre direcciones IP vecinas, aprovechando la correlación espacial inherente en la asignación de bloques IP. El modelo integra características topológicas de red, información de enrutamiento y metadatos de registro para mejorar la precisión de localización. La validación experimental demuestra mejoras significativas sobre métodos tradicionales, especialmente en áreas urbanas densas.

Liu et al.~(2025)~\cite{8} desarrollan EBGeo, un framework que combina redes neuronales convolucionales gráficas con funciones de energía para abordar la incertidumbre en geolocalización IP. Su metodología incorpora modelado probabilístico para quantificar la confianza en las predicciones de ubicación, utilizando técnicas de Monte Carlo para estimación de incertidumbre. Esta aproximación resulta particularmente relevante para aplicaciones de seguridad donde la confiabilidad de la geolocalización es crítica.

\subsection{Plataformas Unificadas de Inteligencia de Amenazas}

Paidy~(2025)~\cite{9} propone una plataforma unificada de detección de amenazas que integra Inteligencia Artificial, SIEM y XDR. La metodología emplea arquitectura distribuida con procesamiento en tiempo real de múltiples fuentes de datos. El sistema utiliza algoritmos de machine learning para correlación de eventos, análisis comportamental para detección de anomalías, y orquestación automatizada de respuestas. Los resultados experimentales demuestran reducción del 40\% en tiempos de respuesta y mejora significativa en visibilidad de amenazas.

Ouaissa et al.~(2025)~\cite{10} desarrollan un framework comprehensivo para modelado de amenazas y evaluación de riesgos en entornos de ciudades inteligentes. Su metodología integra STRIDE con el framework MITRE ATT\&CK para identificación sistemática de amenazas, utiliza diagramas de flujo de datos para visualización de interacciones del sistema, y emplea CVSS junto con matrices de riesgo 5x5 para evaluación quantitativa. El framework incluye estudio de caso específico en Internet de Vehículos utilizando el modelo Cyber Kill Chain para análisis detallado de comportamiento adversarial.

\subsection{Validación de Reputación IP y Análisis de Comportamiento}

Lasantha et al.~(2024)~\cite{11} introducen un framework novel para validación de reputación IP en tiempo real utilizando tecnologías de Inteligencia Artificial. La metodología combina análisis de logs AWS WAF con APIs de reputación como AbuseIPDB, implementa modelos generativos AI para interpretación automatizada de patrones de comportamiento IP, y utiliza algoritmos de machine learning para detección de actividades maliciosas. El sistema incorpora análisis cross-protocolo para detección de IPs maliciosas e implementa mecanismos de respuesta automatizada.

Li et al.~(2024)~\cite{12} abordan el problema de clasificación de escenarios de uso IP mediante un modelo ensemble de árboles neuronales continuos profundos. Su metodología extrae características IP a través de mediciones activas de Internet y bases de datos abiertas, utiliza árboles de decisión diferenciables para aprendizaje de transformaciones interpretables, e implementa ecuaciones diferenciales neuronales ordinarias para modelar dependencias entre capas consecutivas. La validación experimental en cuatro regiones geográficas demuestra alta precisión en clasificación y capacidades de transferencia efectivas.

\subsection{Inteligencia de Amenazas Impulsada por IA}

Kwento~(2025)~\cite{13} investiga tecnologías de Inteligencia Artificial en frameworks de ciberseguridad empresarial mediante análisis exhaustivo de literatura. La metodología emplea algoritmos de machine learning que logran precisión de detección superior al 95\%, métodos de deep learning que mejoran F1-scores hasta 33\% sobre técnicas tradicionales, e integración de datos en tiempo real con analítica comportamental. Los hallazgos demuestran capacidades de identificación de amenazas de 150,000 por minuto y prevención de 8 de cada 10 ataques antes del compromiso del sistema.

\subsection{Dashboards Interactivos y Visualización de Seguridad}

Reddy~(2024)~\cite{14} propone un enfoque unificado para ciberseguridad y seguridad de información dentro de una plataforma única. La metodología integra múltiples dominios de seguridad en sistema cohesivo, implementa dashboards centralizados con flujos automatizados, utiliza IA y ML para análisis predictivo y detección de anomalías, y incorpora cumplimiento regulatorio automatizado. La evaluación demuestra mejoras significativas en gestión de riesgos, procesos de cumplimiento y eficiencia operacional.

Valadez-Godínez et al.~(2021)~\cite{15} desarrollan un dashboard interactivo de ciberseguridad para monitoreo en tiempo real de incidentes. Su metodología utiliza Microsoft Azure APIs para recolección centralizada de datos, implementa Power BI para visualización avanzada y generación de reportes, integra múltiples fuentes de datos mediante JSON para compatibilidad y seguridad, y proporciona filtros personalizados para gestión específica por departamento. La solución demuestra efectividad en centralización de información y mejora en capacidades de respuesta a incidentes.

\subsection{Análisis Comparativo y Contraste Metodológico}

El análisis de estos trabajos revela varias aproximaciones metodológicas predominantes: (1) \textbf{Enfoques basados en machine learning y deep learning} para análisis de patrones y detección de anomalías~\cite{11,13}; (2) \textbf{Arquitecturas distribuidas y procesamiento en tiempo real} para manejo de volúmenes masivos de datos~\cite{6,7,8,9}; (3) \textbf{Integración de múltiples fuentes de datos} para enriquecimiento de contexto~\cite{10,14,15}; (4) \textbf{Modelado probabilístico y quantificación de incertidumbre} para evaluación de confianza; y (5) \textbf{Frameworks híbridos} que combinan técnicas tradicionales con aproximaciones innovadoras.

La mayoría de los trabajos se enfoca en aspectos específicos del problema: geolocalización IP~\cite{6,7,8}, detección de amenazas~\cite{9,12}, o análisis de reputación~\cite{11}. Pocos abordan el problema de manera integral, y ninguno proporciona una solución unificada específicamente diseñada para el contexto colombiano utilizando exclusivamente recursos gratuitos~\cite{13}.

\subsection{Pertinencia para el Proyecto Propuesto}

La propuesta de este proyecto se distingue por su \textbf{enfoque integrador} que combina elementos metodológicos de múltiples trabajos analizados. Específicamente, adopta: (1) \textbf{Técnicas de geolocalización mejoradas} inspiradas en Wang et al.~\cite{7} y Liu et al.~\cite{8}, pero implementadas sobre GeoLite2 local para eliminación de límites de uso; (2) \textbf{Arquitectura unificada de inteligencia de amenazas} similar a Paidy~\cite{9} y Reddy~\cite{14}, pero optimizada para fuentes gratuitas; (3) \textbf{Metodología de scoring de riesgo} que incorpora elementos de Ouaissa et al.~\cite{10} y Lasantha et al.~\cite{11} para evaluación contextual de amenazas; y (4) \textbf{Dashboard interactivo} con características avanzadas de visualización basadas en Valadez-Godínez et al.~\cite{15}.

\subsection{Ventajas de la Metodología Propuesta}

La metodología de este proyecto presenta varias ventajas distintivas: (1) \textbf{Sostenibilidad económica} mediante uso exclusivo de fuentes gratuitas, eliminando barreras de adopción; (2) \textbf{Contextualización regional} específica para Colombia mediante integración con LACNIC y consideración de patrones locales; (3) \textbf{Modelo de scoring integrado} que combina múltiples dimensiones de riesgo en evaluación unificada; (4) \textbf{Arquitectura escalable} diseñada para crecimiento y extensión futura; y (5) \textbf{Enfoque de código abierto} que facilita replicación y mejora comunitaria.

La convergencia de estas características metodológicas posiciona al proyecto como una contribución significativa al campo, particularmente en contextos donde la accesibilidad y sostenibilidad económica son factores críticos para la adopción de soluciones de ciberseguridad.
