\selectlanguage{spanish}
\begin{resumen}
La evaluación de seguridad de direcciones IP requiere actualmente consultar múltiples plataformas dispersas, generando un proceso manual ineficiente y propenso a errores. Este trabajo presenta DiagSEG, una herramienta web de diagnóstico unificado que integra escaneo activo de red, análisis de vulnerabilidades, geolocalización y contexto de red para direcciones IP y dominios. La plataforma utiliza exclusivamente fuentes de datos gratuitas: Nmap con TCP connect scan para detección de servicios, National Vulnerability Database (NVD) de NIST para identificación de CVEs, base de datos GeoLite2 de MaxMind para geolocalización e ipapi.co para información de ASN e ISP. El sistema implementa una arquitectura cloud-native stateless con backend Quarkus 3.29.2 desplegado en Railway y frontend Vue.js 3 desplegado en Vercel, proporcionando alta disponibilidad sin costos de licenciamiento. Se desarrolló un modelo algorítmico de scoring de riesgo que procesa datos heterogéneos para generar evaluaciones integrales con recomendaciones contextuales. La plataforma se encuentra desplegada en producción en \url{https://diagsec.vercel.app}, demostrando la viabilidad de crear herramientas profesionales de ciberseguridad utilizando únicamente tecnologías open source y APIs públicas. Los resultados validan la efectividad del sistema para democratizar el acceso a inteligencia de amenazas cibernéticas en el contexto colombiano.
\end{resumen}

\begin{palabrasclave}
reconocimiento activo, geolocalización IP, inteligencia de amenazas cibernéticas, análisis de vulnerabilidades, arquitectura cloud-native.
\end{palabrasclave}

\selectlanguage{english}
\begin{abstract}
Security assessment of IP addresses currently requires consulting multiple dispersed platforms, creating an inefficient and error-prone manual process. This work presents DiagSEG, a unified web diagnostic tool that integrates active network scanning, vulnerability analysis, geolocation, and network context for IP addresses and domains. The platform exclusively utilizes free data sources: Nmap with TCP connect scan for service detection, NIST's National Vulnerability Database (NVD) for CVE identification, MaxMind's GeoLite2 database for geolocation, and ipapi.co for ASN and ISP information. The system implements a stateless cloud-native architecture with Quarkus 3.29.2 backend deployed on Railway and Vue.js 3 frontend deployed on Vercel, providing high availability without licensing costs. An algorithmic risk scoring model was developed to process heterogeneous data and generate comprehensive assessments with contextual recommendations. The platform is deployed in production at \url{https://diagsec.vercel.app}, demonstrating the feasibility of creating professional cybersecurity tools using only open-source technologies and public APIs. Results validate the system's effectiveness in democratizing access to cyber threat intelligence in the Colombian context.
\end{abstract}

\begin{IEEEkeywords}
active reconnaissance, IP geolocation, cyber threat intelligence, vulnerability analysis, cloud-native architecture.
\end{IEEEkeywords}