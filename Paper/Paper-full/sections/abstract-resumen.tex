\selectlanguage{spanish}
\begin{resumen}
En la actualidad, la evaluación de la legitimidad y seguridad de direcciones IP requiere consultar múltiples plataformas dispersas, generando un proceso manual lento e ineficiente. Este trabajo propone el desarrollo de una herramienta web unificada de diagnóstico que integra datos de escaneo de red, análisis de vulnerabilidades, geolocalización aproximada y contexto de red para direcciones IP. La plataforma utiliza fuentes de datos completamente gratuitas incluyendo Nmap para detección de servicios y puertos abiertos, la National Vulnerability Database (NVD) de NIST para identificación de vulnerabilidades, la base de datos GeoLite2 de MaxMind para geolocalización, y consultas WHOIS a registros regionales. Se implementa un modelo algorítmico de puntuación de riesgo que procesa los datos recolectados para generar una evaluación integral de amenazas. La herramienta democratiza el acceso a inteligencia de amenazas cibernéticas mediante una interfaz web intuitiva que presenta informes consolidados con recomendaciones de seguridad contextuales. Los resultados esperados incluyen una plataforma funcional de acceso público sin dependencias de servicios comerciales y documentación técnica para replicación del proyecto, contribuyendo significativamente a la mejora de las capacidades de diagnóstico de seguridad en Colombia.
\end{resumen}

\begin{palabrasclave}
reconocimiento pasivo, geolocalización IP, inteligencia de amenazas cibernéticas, análisis de seguridad de red, plataforma unificada de diagnóstico.
\end{palabrasclave}

\selectlanguage{english}
\begin{abstract}
Currently, evaluating the legitimacy and security of IP addresses requires consulting multiple dispersed platforms, creating a slow and inefficient manual process. This work proposes the development of a unified web diagnostic tool that integrates network scanning, vulnerability analysis, approximate geolocation, and network context for IP addresses. The platform utilizes completely free data sources including Nmap for service and open port detection, NIST's National Vulnerability Database (NVD) for vulnerability identification, MaxMind's GeoLite2 database for geolocation, and WHOIS queries to regional registries. An algorithmic risk scoring model is implemented to process collected data and generate comprehensive threat assessments. The tool democratizes access to cyber threat intelligence through an intuitive web interface that presents consolidated reports with contextual security recommendations. Expected results include a functional public access platform without dependencies on commercial services and technical documentation for project replication, significantly contributing to improved security diagnostic capabilities in Colombia.
\end{abstract}

\begin{IEEEkeywords}
passive reconnaissance, IP geolocation, cyber threat intelligence, network security analysis, unified diagnostic platform.
\end{IEEEkeywords}
