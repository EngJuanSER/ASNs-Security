\section{Conclusiones}\label{sec:conclusion}

Este trabajo presentó el diseño, implementación y evaluación de DiagSEG, una plataforma web de diagnóstico unificado para análisis de seguridad de direcciones IP que integra escaneo activo de red, análisis de vulnerabilidades, geolocalización y contexto de red. El sistema fue desplegado exitosamente en producción utilizando exclusivamente tecnologías open source y APIs públicas gratuitas, demostrando la viabilidad de crear herramientas profesionales de ciberseguridad sin costos de licenciamiento.

\subsection{Logros Alcanzados}

La investigación cumplió exitosamente sus objetivos, logrando los siguientes resultados:

\textbf{Integración efectiva de fuentes heterogéneas:} El sistema consolidó cuatro fuentes de datos dispares (Nmap, NVD, GeoLite2, ipapi.co) en una plataforma coherente. La arquitectura modular implementada con servicios CDI independientes demostró ser mantenible y extensible.

\textbf{Arquitectura cloud-native stateless funcional:} El despliegue en Railway y Vercel logró 99.5\% de disponibilidad durante 4 semanas de evaluación sin infraestructura dedicada ni costos operacionales. La arquitectura stateless implementada simplifica el escalamiento y operación.

\textbf{Superación de restricciones técnicas:} La adaptación exitosa del escaneo Nmap a TCP connect scan (-sT) resolvió las limitaciones de raw sockets en Railway. El sistema de cache frontend implementado (42\% hit rate) mitiga efectivamente el overhead de latencia adicional.

\textbf{Modelo de scoring operacional:} El algoritmo de evaluación de riesgo implementado procesa efectivamente datos heterogéneos para generar scores normalizados 0-100 con recomendaciones contextuales. Los usuarios evaluaron la claridad de resultados con 4.4/5.0 puntos.

\textbf{Validación con usuarios reales:} Durante 4 semanas, 347 usuarios únicos realizaron 500 análisis, con 28\% de retorno y 87\% expresando disposición a uso regular. El 93\% del tráfico provino de América Latina, confirmando la relevancia regional del sistema.

\subsection{Contribuciones Principales}

Este trabajo aporta las siguientes contribuciones al campo:

\begin{enumerate}
    \item \textbf{Arquitectura de referencia documentada:} Se implementó y documentó una arquitectura cloud-native completa que resuelve el desafío de integrar Nmap en entornos containerizados sin raw sockets. El código fuente público facilita replicación.
    
    \item \textbf{Democratización de capacidades:} Se proporcionó acceso gratuito a inteligencia de amenazas comparable a herramientas comerciales (Shodan \$59/mes, VirusTotal \$42/mes), demostrando que la combinación estratégica de fuentes públicas genera análisis de calidad profesional.
    
    \item \textbf{Validación empírica cuantitativa:} Se proporcionaron métricas detalladas de rendimiento (15.8s análisis promedio, 78\% tiempo en Nmap), precisión de geolocalización (97\% país, 68\% ciudad) y efectividad de cache (42\% hit rate, 98.9\% reducción de latencia).
    
    \item \textbf{Evidencia de adopción orgánica:} Los datos de uso real durante 4 semanas validan empíricamente la necesidad y utilidad del sistema en el contexto latinoamericano objetivo.
\end{enumerate}

\subsection{Implicaciones}

Los resultados tienen implicaciones significativas para el contexto colombiano y latinoamericano:

La disponibilidad de una herramienta gratuita y funcional reduce la dependencia de plataformas comerciales internacionales, contribuyendo a soberanía tecnológica. El 73\% de accesos desde Colombia valida la relevancia local identificada.

El código abierto y documentación exhaustiva facilitan la transferencia de conocimiento a estudiantes y profesionales emergentes en ciberseguridad, contribuyendo a la formación de capacidades técnicas locales.

El proyecto establece un precedente replicable demostrando que es viable desarrollar herramientas especializadas de calidad profesional en contextos académicos sin requerir inversiones en infraestructura o licencias comerciales.

\subsection{Limitaciones}

Es importante reconocer las limitaciones del trabajo realizado:

\textbf{Escalabilidad limitada:} El tier gratuito de Railway soporta aproximadamente 1000 análisis/día. La validación se realizó con cargas menores, por lo que el comportamiento bajo cargas sostenidas mayores no fue evaluado.

\textbf{Cobertura de puertos restringida:} El escaneo de 8 puertos predeterminados es adecuado para análisis rápidos pero insuficiente para auditorías de seguridad exhaustivas. Esta decisión de diseño priorizó UX sobre profundidad de análisis.

\textbf{Validación geográfica concentrada:} La evaluación se realizó principalmente con usuarios colombianos (73\%). La efectividad en otros contextos latinoamericanos no fue validada directamente.

\textbf{Precisión de geolocalización a nivel ciudad:} GeoLite2 alcanzó solo 68\% de precisión a nivel ciudad, limitando la utilidad para casos de uso que requieren localización geográfica precisa.

\subsection{Trabajo Futuro}

El trabajo realizado sugiere direcciones naturales para extensión:

La paralelización de consultas NVD mediante thread pools podría reducir significativamente la latencia total de análisis cuando múltiples servicios requieren correlación de vulnerabilidades.

La automatización de actualizaciones mensuales de GeoLite2 mediante scheduled jobs eliminaría la intervención manual actualmente requerida para mantener la base de datos actualizada.

La implementación de cache distribuido (Redis) podría incrementar el hit rate al compartir resultados entre usuarios, especialmente para IPs frecuentemente consultadas como servicios públicos (Google DNS, Cloudflare).

La integración de fuentes adicionales de threat intelligence (AlienVault OTX, AbuseIPDB) enriquecería el análisis de reputación con indicadores de compromiso validados por la comunidad.

\subsection{Reflexión Final}

Este trabajo demostró empíricamente que es viable crear herramientas profesionales de ciberseguridad utilizando exclusivamente tecnologías open source y servicios cloud gratuitos, sin comprometer calidad ni funcionalidad. La adopción orgánica por 347 usuarios durante el período de evaluación valida la necesidad de soluciones accesibles que democraticen capacidades típicamente reservadas a organizaciones con presupuestos significativos.

El proyecto contribuye al fortalecimiento de capacidades locales de ciberseguridad en Colombia y América Latina, estableciendo un precedente replicable para el desarrollo de herramientas especializadas en contextos académicos y de recursos limitados. La disponibilidad del código fuente completo facilita que futuros investigadores extiendan este trabajo, adaptándolo a sus necesidades específicas.
